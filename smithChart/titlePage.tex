\documentclass{ximera}

\input{../preamble.tex}

\title{Smith Chart}
\author{Milica Markovi{\'c}}

\begin{document}

\begin{abstract}
%Stuff can go here later if we want!
\end{abstract}

\maketitle

\begin{sectionOutcomes}

After completing this section, students should be able to do the following.

\begin{itemize}
\item Describe Smith chart as a polar plot of reflection coefficient.
\item Estimate the reflection coefficient from the Smith Chart.
\item Read the reflection coefficient off the Smith Chart
\item Mark the point on the Smith Chart given reflection coefficient.
\item Calculate load impedance if reflection coefficient and transmission-line impedance are given
\item Calculate reflection coefficient if load impedance and transmission-line impedance are given
\item Explain how was Smith Chart developed
\item Read normalized impedance and reflection coefficient on Smith Chart  given a random point on the Smith Chart
\item Given impedance find the normalized position of the impedance on the Smith Chart.
\item Describe the reason for introducing admittance
\item Write impedance and admittance of an inductor and capacitor. 
\item Explain why is the susceptance of an inductor negative and reactance is positive. 
\item Explain why is the admittance Smith Chart rotated 180 degrees.
\item Distinguish between load impedance and normalized load impedance. Describe impedances on the Smith Chart as normalized impedances
\item Given impedance, read admittance on combo Y/Z chart
\item Given a random point on Y/Z chart, find admittance, impedance and the reflection coefficient.
\item Explain electrical length
\item Calculate the input impedance and input reflection coefficient
\item Describe input reflection coefficient in terms of load reflection coefficient.
\end{itemize}

\end{sectionOutcomes}

\end{document}
