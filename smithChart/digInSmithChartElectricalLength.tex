\documentclass{ximera}  


\input{../preamble.tex}



 
\title{Electrical Length} 
\author{Milica Markovic} 
\outcome{}
\begin{document}  
\begin{abstract}  

\end{abstract}  
\maketitle    





\section{Electrical Length of the line in meters}

We can express the physical length of the line in meters. However, in high-frequency electronics (microwave engineering), we usually convert this length to the fraction of a wavelength of a signal that is traveling on the line. 


\begin{eqnarray}
 l=N \lambda 
\end{eqnarray}
 

Typically, but not always, $N$ is a fraction, for example,  $N=\frac{1}{2}=0.5$, $N=\frac{1}{4}=0.25$ or $N=\frac{1}{8}=0.125$; although it can be any  The lenght of the line is then written as


\begin{eqnarray}
 l=\frac{\lambda}{2}  \\
 l\frac{\lambda}{4}  \\
  l=\frac{\lambda}{8}  
\end{eqnarray}
 


If the physical length of the line is $l=\frac{\lambda}{4}$, we say: “This line is quarter-wavelength long at 1GHz”, meaning one-quarter of the wavelength fits on the line. We could also say that the line is 7.5cm long, as wavelength is $\lambda=30 \unit{cm}$ at $1$GHz.


When we say quarter-wavelength long, we refer to the line’s physical length at a specific frequency. 

\section{Electrical length of the line in degrees}

The phase shift between input and output signal on a transmission line is  
$\Theta=\beta*l$.
$beta$ is called the phase constant. It represents the spatial frequency of the signal.
$\Theta=\beta*L$ is the phase in degrees or radians (related is a time delay in seconds).
$\Theta$  can be, for example $\Theta = 45^0$, $\Theta = 90^0$, $\Theta = 180^0$. $\Theta$ is a function of frequency, because $\beta$ is a function of frequency.
 If $\Theta = 90^0$, we say: “The line is 90 degrees long at 1GHz”, meaning the output signal at 1GHz will be shifted for $90^0$ with respect to the input signal. 
When we say $90^0$, we refer to the line’s electrical length, representing the number of degrees that the line introduces between the input and the output signal.


\begin{example}
What is the electrical length of a 30\,cm line in terms of the fraction of wavelength at 1\,GHz? What is the electrical length of the line at 1GHz?
\begin{explanation}
Wavelength at 1\,GHz, assuming the wave is propagating in air is $\lambda=\frac{c}{f}=30$\,cm. Since the line is also l=30\,cm long, the length of the line in terms of wavelength is $\frac{l}{\lambda}=1$, or $l=\lambda$.

The electrical length of the line is $\theta=\beta l = \frac{2 \pi}{\lambda} \lambda= 2 \pi = 360^0$.

\end{explanation}

\end{example}


\begin{example}
What is the electrical length of a 15\,cm line in terms of the fraction of wavelength at 1\,GHz? What is the electrical length of the line at 1GHz?
\begin{explanation}
Wavelenght at 1\,GHz, assuming the wave is propagating in air is $\lambda=\frac{c}{f}=30$\,cm. Since the line is 15\,cm long, the length of the line in terms of wavelenth is $\frac{l}{\lambda}=\frac{1}{2}$, or $l=\frac{\lambda}{2}$.

The electrical length of the line is $\theta=\beta l = \frac{2 \pi}{\lambda} \frac{\lambda}{2} =  \pi = 180^0$.


\end{explanation}

\end{example}


\begin{example}
What is the electrical length of a 7.5\,cm line in terms of the fraction of wavelength at 1\,GHz? What is the electrical length of the line at 1GHz?
\begin{explanation}
Wavelength at 1\,GHz, assuming the wave is propagating in air is $\lambda=\frac{c}{f}=30$\,cm. Since the line is 7.5\,cm long, the line's length in terms of wavelenth is $l=\frac{\lambda}{4}$.

The electrical length of the line is $\theta=\beta l = \frac{2 \pi}{\lambda} \frac{\lambda}{4} =  \pi/2 = 90^0$.


\end{explanation}

\end{example}


\end{document} 
