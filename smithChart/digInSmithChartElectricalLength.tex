\documentclass{ximera}  


\input{../preamble.tex}



 
\title{Electrical Length} 
\author{Milica Markovic} 
\outcome{}
\begin{document}  
\begin{abstract}  

\end{abstract}  
\maketitle    





\section{Electrical Length of the line in meters}

We can express the physical lenght of the line in meters. However in high-frequency electronics (microwave engineering) we usually convert this length to the fraction of a wavelength of a signal that is travelling on the line. 


\begin{eqnarray}
 L=N \lambda
\end{eqnarray}
 

Typically, but not always, $N$ is a fraction, for example,  $N=\frac{1}{2}=0.5$, $N=\frac{1}{4}=0.25$ or $N=\frac{1}{8}=0.125$; although it can be any  The lenght of the line is then written as


\begin{eqnarray}
 L=\frac{\lambda}{2}  \\
 L=\frac{\lambda}{4}  \\
  L=\frac{\lambda}{8}  
\end{eqnarray}
 


If physical lenght of the line is $L=\frac{\lambda}{4}$, we say: “This line is quarter-wavelength long at 1GHz”, meaning one quarter of the wavelength fits on the line. We could also say that the line is 7.5cm long, as wavelength is $\lambda=30 \unit{cm}$ at $1$GHz.


When we say quarter-wavelength long, we refer to line’s physical length. To make things a little more confusing,

\section{Electrical length of the line in degrees}

Phase shift between input and output signal on a transmission line is  
$\Theta=\beta*L$.
$beta$ is called the phase constant. It represents the spatial frequency of signal.
$\Theta=\beta*L$ is the phase in degrees or radians (related is time delay in seconds).
$\Theta$  can be, for example $\Theta = 45^0$, $\Theta = 90^0$, $\Theta = 180^0$. $\Theta$ is a function of frequency, because $\beta$ is a function of frequency.
 If $\Theta = 90^0$ we say: “The line is 90 degrees long at 1GHz”, meaning the output signal at 1GHz will be shifted for $90^0$ with respect to the input signal. 
When we say $90^0$ we refer to line’s electrical length


\end{document} 
