\documentclass{ximera}  


\input{../preamble.tex}



 
\title{Force on Conductors} 
\author{Milica Markovic} 
\outcome{Forces on Conductors in Magnetic Fields.}
\begin{document}  
\begin{abstract}  

\end{abstract}  
\maketitle    



\section{Force on a loop of current due to external magnetic field}


Figure \ref{fig:currentLoop} shows a rectangular 


\begin{figure}[htbp]
\begin{center}
\includegraphics[scale=0.5]{../jpg/loop.jpg}
\end{center}
\caption{Current loop in external magnetic field. }
\label{fig:currentLoop}
\end{figure}


Force $\vec{F}$ on a conductor of length $\vec{l}$ due to an external magentic field $\vec{B}$ is

\begin{equation}
\vec{F}=I \vec{l} \cross \vec{B}
\end{equation}

Vector $\vec{l}$ is in the direction of the current.

\section{DC Motor}



\begin{figure}[htbp]
\begin{center}
\includegraphics[scale=0.5]{../jpg/dcMotor.jpg}
\end{center}
\caption{Simplified schematic of a DC motor. }
\label{fig:dcMotor}
\end{figure}

\section{Force between two loops of current}

When a conductor of lenght $l_1$, carrying current $I_1$, is in a vicinity of another conductor $l_2$, carrying current $I_2$, the force acts between them 


\begin{eqnarray}
\vec{F_1}=I_1 \vec{l_1} \cross \vec{B_2} \\
\vec{F_2}=I_2 \vec{l_2} \cross \vec{B_1}
\end{eqnarray}

Where $F_1$ is the force on conductor $l_1$ $B_1$ is the magnetic field due to current $I_1$ in conductor $l_1$, and $B_2$ is the magnetic field due to current $I_2$.

To find the direction of the force, you have to use the right hand rule. Here is a simulation that shows the force when the currents are in the same direction. What is the direction of the force?

\begin{center}  
\youtube{fz385mDNj84}  
\end{center} 

This video shows a simulation that shows the force when the currents are in the opposite direction. What is the direction of the force now?


\begin{center}  
\youtube{P49PHzqIS8s}  
\end{center} 




\end{document} 
