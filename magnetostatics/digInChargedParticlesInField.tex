\documentclass{ximera}  


\input{../preamble.tex}



 
\title{Charged particles in static electric and magnetic fields} 
\author{Milica Markovic} 
\outcome{Charged Particles in a static electric and magentic field.}
\begin{document}  
\begin{abstract}  

\end{abstract}  
\maketitle    

\section{Cross product}

Cross product is defined as

\begin{equation}
\vec{C}=\vec{A} \cross \vec{B}
\end{equation}

To find the direction of the vector $\vec{C}$, use the right-hand rule. Place the first vector in the cross product $\vec{A}$ in the palm of your right hand, and the second vector $\vec{B}$ sticking out perpendiculary to your palm. Then sweep the fingers from vector $\vec{A}$ to $\vec{B}$ and your thumb will be pointing to the direction of the vector $\vec{C}$.


\begin{figure}[htbp]
\begin{center}
\includegraphics[scale=0.5]{../jpg/RHR.jpg}
\end{center}
\caption{Using right-hand rule to find the direction of cross product. }
\label{fig:crossProduct}
\end{figure}

\section{Magnetic Force on a charged particle}

Magnetic force on a charged particle $F=q \vec{v} \cross \vec{B}$ will change the direction of the particle's path, perpendicularly to the direction of the magnetic field and the velocity vector, but it will not change the speed of the particle. The magnetic field cannot accelerate the charged particle. Use the applet below to change the charge, mass, and velocity of the particle to see how the path of the particle changes.


\geogebra{HXP8xUC3}{800}{600}
%\geogebra{xpRMzPgc}{1300}{1000}



\section{Lorenz Force}

The force on a charged particle in electric and magnetic field, a Lorentz force, is  

\begin{equation}
\vec{F}=q\vec{E} + q \vec{v} \cross \vec{B}
\end{equation}

Where $q$ is the charge of the particle, $\vec{E}$ is external electric field, $v$ is the velocity of the particle and $\vec{B}$ is the magnetic field.


Here is an interesting applet that let's you see the charged particle's path by setting the strength of electric and magnetic field, charge and mass of the particle, and the speed direction and magnitude.

%The applet is too slow, need to change it to speed it up, eg remove stuff
%\geogebra{J9nrrpQH}{1300}{1000}





\end{document} 
