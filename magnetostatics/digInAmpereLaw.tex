\documentclass{ximera}  


\input{../preamble.tex}



 
\title{Ampere's Law} 
\author{Milica Markovic} 
\outcome{Ampere's Law}
\begin{document}  
\begin{abstract}  

\end{abstract}  
\maketitle    


\section{Static Magnetic Field}

Source of static magnetic fields are DC currents and magnets. The direction of the magnetic field from a DC current is determined through right-hand rule. The direction of the magnetic field from the magnet is outward oriented from the north pole and inward into the south pole.

It is interesting to see that the magnetic field from a loop of current looks very much like a magnetic field of a magnet placed in the center of the loop, perpendicularly to it. 

Geographical earth's south pole is earth's magnetic north pole.


\subsection{Ampere's Law}


Ampere's law for static magnetic fields states that the integral of the magnetic field around closed contour is equal to the current enclosed by the contour.

\begin{equation}
\oint_c \vec{H} \cdot \vec{dl} = I
\end{equation}


If the contour encloses only part of the current, whose current density is J, then


\begin{equation}
\oint_c \vec{H} \cdot \vec{dl} = \int_S \vec{J} \cdot \vec{dS}
\end{equation}


The vector $\vec{dS}$ is the surface area enclosed by the contour. The vector of the surface area is determined from the direction of the contour with the right-hand rule. If we orieng our fingers in the direction of the contour, the thumb will point in the direction of surface vector.







\begin{example}
\subsection{Magnetic field of an infinite wire with circular cross section}





\begin{figure}[htbp]
\begin{center}
\includegraphics[scale=0.5]{../jpg/Ampere's_Law1.jpg}
\end{center}
\caption{Application of Ampere's law outside an infinite straight conductor}
\label{MutualInduc}
\end{figure}




\begin{figure}[htbp]
\begin{center}
\includegraphics[scale=0.5]{../jpg/Ampere's_Law2.jpg}
\end{center}
\caption{Application of Ampere's law inside an infinite straight conductor}
\label{MutualInduc}
\end{figure}




\end{example}

\begin{example}
\subsection{Magnetic field of a solenoid}

\end{example}


\begin{example}
\subsection{Magnetic field of a torroid}

\end{example}


\end{document} 
