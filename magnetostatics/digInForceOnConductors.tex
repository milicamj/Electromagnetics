\documentclass{ximera}  


\input{../preamble.tex}



 
\title{Force on Conductors} 
\author{Milica Markovic} 
\outcome{Forces on Conductors in Magnetic Fields.}
\begin{document}  
\begin{abstract}  

\end{abstract}  
\maketitle    



\section{Force on a conductor due to external magentic field}


Force $\vec{F}$ on a conductor of length $\vec{l}$ due to an external magentic field $\vec{B}$ is

\begin{equation}
\vec{F}=I \vec{l} \cross \vec{B}
\end{equation}

Vector $\vec{l}$ is in the direction of the current.


\section{Direction of a manetic field of a conductor}

The direction of a magnetic field of a conductor can be determined using the right-hand rule. If you point the thumb of your right hand in the direction of the current in a straight conductor, your fingers will curl in the direction of the magnetic field.

 To see how the magnetic field changes around a straight conductor, change the strenght and the direction of current in the applet below. The distance between the magnetic field rings represents the strenght of the magnetic field. When the rings are closer, the magnetic field is stronger, and vice versa. You can also see the direction of current as the vertical arrow and a small horizontal arrow shows the direction of the magnetic field. 


\begin{center}  
\geogebra{Xr65rR3b}{800}{600}  
\end{center} 


\section{Force between two conductors}

When a conductor of lenght $l_1$, carrying current $I_1$, is in a vicinity of another conductor $l_2$, carrying current $I_2$, the force acts between them 


\begin{eqnarray}
\vec{F_1}=I_1 \vec{l_1} \cross \vec{B_2} \\
\vec{F_2}=I_2 \vec{l_2} \cross \vec{B_1}
\end{eqnarray}

Where $F_1$ is the force on conductor $l_1$ $B_1$ is the magnetic field due to current $I_1$ in conductor $l_1$, and $B_2$ is the magnetic field due to current $I_2$.

To find the direction of the force, you have to use the right hand rule. Here is a simulation that shows the force when the currents are in the same direction. What is the direction of the force?

\begin{center}  
\youtube{fz385mDNj84}  
\end{center} 

This video shows a simulation that shows the force when the currents are in the opposite direction. What is the direction of the force now?


\begin{center}  
\youtube{P49PHzqIS8s}  
\end{center} 




\end{document} 
