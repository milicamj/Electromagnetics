\documentclass{ximera}

\input{../preamble.tex}

\title{Complex numbers}
\author{Milica Markovi{\'c}}
\begin{document}

\begin{abstract}
%Stuff can go here later if we want!
\end{abstract}

\maketitle

\begin{sectionOutcomes}

After completing this section, students should be able to do the following.

\begin{itemize}
\item Explain why complex numbers are important in circuits and electromagnetics.
\item Sketch a complex number in rectangular and polar coordinates, and label magnitude, phase, real and imaginary parts.
\item Derive the magnitude and phase from the real and imaginary parts of a complex number.
\item Derive the real and imaginary parts of a complex number from the magnitude and phase.
\item Explain how Euler's formula relates sinusoidal signals and complex numbers.
\item Describe which coordinate system to use when adding/subtracting and which one when multiplying/dividing two complex numbers.
\item Apply complex numbers to solve for impedance of a circuit element if phasor of current through and voltage on it are known.
\item Apply complex numbers to solve for voltage on a circuit element if phasor of current and impedance are known.
\item Apply complex conjugate operation to a complex number in rectangular and polar coordinates.
\item Derive magnitude of a complex number from a complex number and complex conjugate of the same number.
\item Visualize the position of  purely imaginary and purely real complex numbers on a unit circle.
\item Convert visually purely imaginary and purely real complex numbers from rectangular to polar coordinates and vice versa.
\item Prove that the magnitude of a complex number is a square root of the product of the number and its complex conjugate.
\end{itemize}

\end{sectionOutcomes}

\end{document}
