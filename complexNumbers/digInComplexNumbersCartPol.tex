\documentclass{ximera}  

\input{../preamble.tex}


 
\title{Review of Complex Numbers} 
\author{Milica Markovic} 
\outcome{Sketch a complex number in rectangular and polar coordinates, and label magnitude, phase, real and imaginary parts. Derive the magnitude and phase from the real and imaginary parts of a complex number. Derive the real and imaginary parts of a complex number from the magnitude and phase.}
\begin{document}  
\begin{abstract}  
The purpose of this section is to introduce two different ways we represent complex nubers: Cartesian coordinates, and Polar coordinates. We often use complex numbers in Cartesian coordinates when we discuss impedances, or admittances. We often use complex numbers in polar coordinates to discuss magnitude and phase of voltages, currents, transfer functions, and Bode Plots. Sinusoidal signals are also represented with complex numbers with phasors. It is critically important that we understand this chapter.
\end{abstract}  
\maketitle    
  
\begin{definition}
 A complex number  $z$ can be represented in Cartesian coordinate system as shown in Equation \ref{cartesian}, and  Polar coordinate system as shown in Equation \ref{polar}.




\begin{eqnarray}
z= x + j y \label{cartesian} \\ 
z=|z| e^{j \Theta} \label{polar}
\end{eqnarray}

\end{definition}

In Equation \ref{cartesian}, a complex number z is represented in rectangular coordinate system, where $x$ is the real part, $y$ is the imaginary part, and $j=\sqrt{-1}$. 

In Equation \ref{polar}, a complex number z is represented in polar coordinate system, where  $|z|$ is the magnitude and $\Theta$ is the angle (aka phase) of the complex number.

Geometric interpretation of these two equations is given in Figure \ref{cartpol}. The magnitude is the length of the hypothenuse of the triangle shown, and the angle is the angle that the hypothenuse makes with x-axis. 

 As you can see in Figure \ref{cartpol}, we can also represent a complex number with a "position vector". Position vectors are vectors that start at the center of the coordinate system and end at any point in the coordiate system. This is a graphical representation of a phasor. We see that phasor is a vector that represents a complex number in polar coordinate system. 



\begin{figure}[htbp]
\begin{center}
\includegraphics[scale=0.3]{../jpg/Complex_Numbersz12.jpg}
%\strut\psfig{figure=complexnumberz.ps,width=3cm} \\
\end{center}
\caption{Visual representation of a complex number z in rectangular $z=x+jy$ and polar coordinates $z=|z|e^{j \theta}$.}
\label{cartpol}
\end{figure}




You may be wondering why are we representing the phase of a complex number in polar coordinate system as  $e^{j \Theta}$, because in the circuits class you used $\angle \theta$. Great question. That brings us to Eurler's formula that we will discuss in the section Euler's Formula.

\section{Conversion between Cartesian and Polar coordinate systems}
  
  To find magnitude and angle when you know real and imaginary part of a complex number, use the Pythagorean Theorem to find the magnitude of the complex number as in Equation \ref{eq:polar_mag}, and use the definition of tangent to find the angle as in Equation \ref{eq:polar_angle}.

\begin{eqnarray}
|z|=\sqrt{x^2+y^2} \label{eq:polar_mag}\\
\Theta = arctg \frac{y}{x} \label{eq:polar_angle}
\end{eqnarray}
 
 
 To find the real and imaginary part of a complex number when you know magnitude and phase, use trigonometry. To find the real part of the complex number as in Equation \ref{eq:cart_real} use the definition of cos, and to find the imaginary part of the complex number use sin, as in Equation \ref{eq:cart_imag}.


\begin{eqnarray}
\Re \{z\} =x= r \cos (\Theta) \label{eq:cart_real}\\
\Im \{z\} =y= r \sin (\Theta)  \label{eq:cart_imag}
\end{eqnarray}


\begin{question}
Explore the conversion of complex numbers between cartesian and polar coordinates.
\begin{center}  
\geogebra{b8hu8ztx}{800}{600}  
\end{center} 
\end{question}




\begin{example}
Find the magnitude and phase of a complex numbers $z_1=j$ and $z_2=1$.


\begin{explanation}

Complex number $z_1=j$ is on the y-axis where y=1. By inspection, the magnitude 
     of $z_1$ is $|z|=1$, and the angle is $\theta=90^o$.



  \begin{image}
\begin{tikzpicture}
	\begin{axis}[
            xmin=-1.1,xmax=1.1,ymin=-1.1,ymax=1.1,
            axis lines=center,
            width=4in,
            xtick={-1,1},
            ytick={-1,1},
            clip=false,
            unit vector ratio*=1 1 1,
            xlabel=$x$, ylabel=$y$,
            every axis y label/.style={at=(current axis.above origin),anchor=south},
            every axis x label/.style={at=(current axis.right of origin),anchor=west},
          ]        
        %  \addplot [dashed, smooth, domain=(0:360)] ({cos(x)},{sin(x)}); %% unit circle

          \addplot [textColor] plot coordinates {(0,0) (0,1)}; %% 90 degrees

          %vertical red up line from 0,0 to 0,1
          \addplot [ultra thick,penColor2] plot coordinates {(0,0) (0,1)}; %% 90 degrees
          
         % draw small circle to designate that theta is 90 degrees        

          \addplot [textColor,smooth, domain=(0:90)] ({.15*cos(x)},{.15*sin(x)});
          % write theta next to the small circle
         
          \node at (axis cs:.11,.11) [anchor=west] {$\theta$};
          %write j next to the y=1
         
          \node at (axis cs:0.35,1) [anchor=east] {$z=j$};
          
        \end{axis}
\end{tikzpicture}
\end{image}
  

Complex plane is sketched below.
Complex number $z_1=1$ is on the x-axis where x=1. By inspection, the magnitude 
     of $z_1$ is $|z|=1$, and the angle is $\theta=0^o$.


  \begin{image}
\begin{tikzpicture}
	\begin{axis}[
            xmin=-1.1,xmax=1.1,ymin=-1.1,ymax=1.1,
            axis lines=center,
            width=4in,
            xtick={-1,1},
            ytick={-1,1},
            clip=false,
            unit vector ratio*=1 1 1,
            xlabel=$x$, ylabel=$y$,
            every axis y label/.style={at=(current axis.above origin),anchor=south},
            every axis x label/.style={at=(current axis.right of origin),anchor=west},
          ]        
        %  \addplot [dashed, smooth, domain=(0:360)] ({cos(x)},{sin(x)}); %% unit circle

          \addplot [textColor] plot coordinates {(0,0) (0,1)}; %% 90 degrees

          %horizontal red up line from 0,0 to 0,1
          \addplot [ultra thick,penColor2] plot coordinates {(0,0) (1,0)}; %% 0 degrees
          
         % draw small circle to designate that theta is 90 degrees        

      %    \addplot [textColor,smooth, domain=(0:90)] ({.15*cos(x)},{.15*sin(x)});
          % write theta next to the small circle
         
    %      \node at (axis cs:.11,.11) [anchor=west] {$\theta$};
          %write j next to the y=1
         
          \node at (axis cs:1,0) [anchor=south] {$z=1$};
          
        \end{axis}
\end{tikzpicture}
\end{image}


\end{explanation}


\end{example}


\begin{question}
Calculate magnitude and phase of 
complex number $z=-j$ 
\begin{multipleChoice}  
\choice{$|z|= 1, \theta={180}$}
\choice[correct]{$|z|= 1, \theta={-90}$}   
\choice{$|z|= -1, \theta={180}$}
\choice{$|z|= -1,  \theta={-90}$}
\end{multipleChoice}
\end{question}
  
  
  
  
  
  
  
  
  
  
  
  
  
  
  
  
  
  
  
  
  
  
  
  Write magnitude and phase of complex number $z=-1$
\begin{question}  
         $ -1 =  \answer{e^{j180}}$  
    \end{question} 
    
    
    
    
    
    
    
    
    
    
    
    
    
    
    
    
    
    
    
\end{document} 
