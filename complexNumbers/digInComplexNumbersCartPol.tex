\documentclass{ximera}  

%\usepackage{todonotes}
%\usepackage{mathtools} %% Required for wide table Curl and Greens
%\usepackage{cuted} %% Required for wide table Curl and Greens
\newcommand{\todo}{}

\usepackage{esint} % for \oiint
\ifxake%%https://math.meta.stackexchange.com/questions/9973/how-do-you-render-a-closed-surface-double-integral
\renewcommand{\oiint}{{\large\bigcirc}\kern-1.56em\iint}
\fi


\graphicspath{
  {./}
  {jpg/}
  {ximeraTutorial/}
  {basicPhilosophy/}
  {functionsOfSeveralVariables/}
  {normalVectors/}
  {lagrangeMultipliers/}
  {vectorFields/}
  {greensTheorem/}
  {shapeOfThingsToCome/}
  {dotProducts/}
  {partialDerivativesAndTheGradientVector/}
  {../productAndQuotientRules/exercises/}
  {../motionAndPathsInSpace/exercises/}
  {../normalVectors/exercisesParametricPlots/}
  {../continuityOfFunctionsOfSeveralVariables/exercises/}
  {../partialDerivativesAndTheGradientVector/exercises/}
  {../directionalDerivativeAndChainRule/exercises/}
  {../commonCoordinates/exercisesCylindricalCoordinates/}
  {../commonCoordinates/exercisesSphericalCoordinates/}
  {../greensTheorem/exercisesCurlAndLineIntegrals/}
  {../greensTheorem/exercisesDivergenceAndLineIntegrals/}
  {../shapeOfThingsToCome/exercisesDivergenceTheorem/}
  {../greensTheorem/}
  {../shapeOfThingsToCome/}
  {../separableDifferentialEquations/exercises/}
  {vectorFields/}
}

\newcommand{\mooculus}{\textsf{\textbf{MOOC}\textnormal{\textsf{ULUS}}}}

\usepackage{pgfplots}
% \pgfplotsset{compat=1.15}

\usepackage{tkz-euclide}\usepackage{tikz}
\usepackage{tikz-cd}
\usetikzlibrary{arrows}
\tikzset{>=stealth,commutative diagrams/.cd,
  arrow style=tikz,diagrams={>=stealth}} %% cool arrow head
\tikzset{shorten <>/.style={ shorten >=#1, shorten <=#1 } } %% allows shorter vectors

\usetikzlibrary{backgrounds} %% for boxes around graphs
\usetikzlibrary{shapes,positioning}  %% Clouds and stars
\usetikzlibrary{matrix} %% for matrix
\usepgfplotslibrary{polar} %% for polar plots
\usepgfplotslibrary{fillbetween} %% to shade area between curves in TikZ
\usetkzobj{all}
\usepackage[makeroom]{cancel} %% for strike outs
%\usepackage{mathtools} %% for pretty underbrace % Breaks Ximera
%\usepackage{multicol}
\usepackage{pgffor} %% required for integral for loops



%% http://tex.stackexchange.com/questions/66490/drawing-a-tikz-arc-specifying-the-center
%% Draws beach ball
\tikzset{pics/carc/.style args={#1:#2:#3}{code={\draw[pic actions] (#1:#3) arc(#1:#2:#3);}}}



\usepackage{array}
\setlength{\extrarowheight}{+.1cm}
\newdimen\digitwidth
\settowidth\digitwidth{9}
\def\divrule#1#2{
\noalign{\moveright#1\digitwidth
\vbox{\hrule width#2\digitwidth}}}





\newcommand{\RR}{\mathbb R}
\newcommand{\R}{\mathbb R}
\newcommand{\N}{\mathbb N}
\newcommand{\Z}{\mathbb Z}

\newcommand{\sagemath}{\textsf{SageMath}}


%\renewcommand{\d}{\,d\!}
\renewcommand{\d}{\mathop{}\!d}
\newcommand{\dd}[2][]{\frac{\d #1}{\d #2}}
\newcommand{\pp}[2][]{\frac{\partial #1}{\partial #2}}
\renewcommand{\l}{\ell}
\newcommand{\ddx}{\frac{d}{\d x}}

\newcommand{\zeroOverZero}{\ensuremath{\boldsymbol{\tfrac{0}{0}}}}
\newcommand{\inftyOverInfty}{\ensuremath{\boldsymbol{\tfrac{\infty}{\infty}}}}
\newcommand{\zeroOverInfty}{\ensuremath{\boldsymbol{\tfrac{0}{\infty}}}}
\newcommand{\zeroTimesInfty}{\ensuremath{\small\boldsymbol{0\cdot \infty}}}
\newcommand{\inftyMinusInfty}{\ensuremath{\small\boldsymbol{\infty - \infty}}}
\newcommand{\oneToInfty}{\ensuremath{\boldsymbol{1^\infty}}}
\newcommand{\zeroToZero}{\ensuremath{\boldsymbol{0^0}}}
\newcommand{\inftyToZero}{\ensuremath{\boldsymbol{\infty^0}}}



\newcommand{\numOverZero}{\ensuremath{\boldsymbol{\tfrac{\#}{0}}}}
\newcommand{\dfn}{\textbf}
%\newcommand{\unit}{\,\mathrm}
\newcommand{\unit}{\mathop{}\!\mathrm}
\newcommand{\eval}[1]{\bigg[ #1 \bigg]}
\newcommand{\seq}[1]{\left( #1 \right)}
\renewcommand{\epsilon}{\varepsilon}
\renewcommand{\phi}{\varphi}


\renewcommand{\iff}{\Leftrightarrow}

\DeclareMathOperator{\arccot}{arccot}
\DeclareMathOperator{\arcsec}{arcsec}
\DeclareMathOperator{\arccsc}{arccsc}
\DeclareMathOperator{\si}{Si}
\DeclareMathOperator{\scal}{scal}
\DeclareMathOperator{\sign}{sign}


%% \newcommand{\tightoverset}[2]{% for arrow vec
%%   \mathop{#2}\limits^{\vbox to -.5ex{\kern-0.75ex\hbox{$#1$}\vss}}}
\newcommand{\arrowvec}[1]{{\overset{\rightharpoonup}{#1}}}
%\renewcommand{\vec}[1]{\arrowvec{\mathbf{#1}}}
\renewcommand{\vec}[1]{{\overset{\boldsymbol{\rightharpoonup}}{\mathbf{#1}}}\hspace{0in}}

\newcommand{\point}[1]{\left(#1\right)} %this allows \vector{ to be changed to \vector{ with a quick find and replace
\newcommand{\pt}[1]{\mathbf{#1}} %this allows \vec{ to be changed to \vec{ with a quick find and replace
\newcommand{\Lim}[2]{\lim_{\point{#1} \to \point{#2}}} %Bart, I changed this to point since I want to use it.  It runs through both of the exercise and exerciseE files in limits section, which is why it was in each document to start with.

\DeclareMathOperator{\proj}{\mathbf{proj}}
\newcommand{\veci}{{\boldsymbol{\hat{\imath}}}}
\newcommand{\vecj}{{\boldsymbol{\hat{\jmath}}}}
\newcommand{\veck}{{\boldsymbol{\hat{k}}}}
\newcommand{\vecl}{\vec{\boldsymbol{\l}}}
\newcommand{\uvec}[1]{\mathbf{\hat{#1}}}
\newcommand{\utan}{\mathbf{\hat{t}}}
\newcommand{\unormal}{\mathbf{\hat{n}}}
\newcommand{\ubinormal}{\mathbf{\hat{b}}}

\newcommand{\dotp}{\bullet}
\newcommand{\cross}{\boldsymbol\times}
\newcommand{\grad}{\boldsymbol\nabla}
\newcommand{\divergence}{\grad\dotp}
\newcommand{\curl}{\grad\cross}
%\DeclareMathOperator{\divergence}{divergence}
%\DeclareMathOperator{\curl}[1]{\grad\cross #1}
\newcommand{\lto}{\mathop{\longrightarrow\,}\limits}

\renewcommand{\bar}{\overline}

\colorlet{textColor}{black}
\colorlet{background}{white}
\colorlet{penColor}{blue!50!black} % Color of a curve in a plot
\colorlet{penColor2}{red!50!black}% Color of a curve in a plot
\colorlet{penColor3}{red!50!blue} % Color of a curve in a plot
\colorlet{penColor4}{green!50!black} % Color of a curve in a plot
\colorlet{penColor5}{orange!80!black} % Color of a curve in a plot
\colorlet{penColor6}{yellow!70!black} % Color of a curve in a plot
\colorlet{fill1}{penColor!20} % Color of fill in a plot
\colorlet{fill2}{penColor2!20} % Color of fill in a plot
\colorlet{fillp}{fill1} % Color of positive area
\colorlet{filln}{penColor2!20} % Color of negative area
\colorlet{fill3}{penColor3!20} % Fill
\colorlet{fill4}{penColor4!20} % Fill
\colorlet{fill5}{penColor5!20} % Fill
\colorlet{gridColor}{gray!50} % Color of grid in a plot

\newcommand{\surfaceColor}{violet}
\newcommand{\surfaceColorTwo}{redyellow}
\newcommand{\sliceColor}{greenyellow}




\pgfmathdeclarefunction{gauss}{2}{% gives gaussian
  \pgfmathparse{1/(#2*sqrt(2*pi))*exp(-((x-#1)^2)/(2*#2^2))}%
}


%%%%%%%%%%%%%
%% Vectors
%%%%%%%%%%%%%

%% Simple horiz vectors
\renewcommand{\vector}[1]{\left\langle #1\right\rangle}


%% %% Complex Horiz Vectors with angle brackets
%% \makeatletter
%% \renewcommand{\vector}[2][ , ]{\left\langle%
%%   \def\nextitem{\def\nextitem{#1}}%
%%   \@for \el:=#2\do{\nextitem\el}\right\rangle%
%% }
%% \makeatother

%% %% Vertical Vectors
%% \def\vector#1{\begin{bmatrix}\vecListA#1,,\end{bmatrix}}
%% \def\vecListA#1,{\if,#1,\else #1\cr \expandafter \vecListA \fi}

%%%%%%%%%%%%%
%% End of vectors
%%%%%%%%%%%%%

%\newcommand{\fullwidth}{}
%\newcommand{\normalwidth}{}



%% makes a snazzy t-chart for evaluating functions
%\newenvironment{tchart}{\rowcolors{2}{}{background!90!textColor}\array}{\endarray}

%%This is to help with formatting on future title pages.
\newenvironment{sectionOutcomes}{}{}



%% Flowchart stuff
%\tikzstyle{startstop} = [rectangle, rounded corners, minimum width=3cm, minimum height=1cm,text centered, draw=black]
%\tikzstyle{question} = [rectangle, minimum width=3cm, minimum height=1cm, text centered, draw=black]
%\tikzstyle{decision} = [trapezium, trapezium left angle=70, trapezium right angle=110, minimum width=3cm, minimum height=1cm, text centered, draw=black]
%\tikzstyle{question} = [rectangle, rounded corners, minimum width=3cm, minimum height=1cm,text centered, draw=black]
%\tikzstyle{process} = [rectangle, minimum width=3cm, minimum height=1cm, text centered, draw=black]
%\tikzstyle{decision} = [trapezium, trapezium left angle=70, trapezium right angle=110, minimum width=3cm, minimum height=1cm, text centered, draw=black]



 
\title{Review of Complex Numbers} 
\author{Milica Markovic} 
\outcome{Sketch a complex number in rectangular and polar coordinates, and label magnitude, phase, real and imaginary parts. Derive the magnitude and phase from the real and imaginary parts of a complex number. Derive the real and imaginary parts of a complex number from the magnitude and phase.}

\begin{document}  
\begin{abstract}  
This section aims to introduce two different ways we represent complex numbers: Cartesian coordinates, and Polar coordinates. We often use complex numbers in Cartesian coordinates when we discuss impedance or admittance. We often use complex numbers in polar coordinates to discuss magnitude and phase of voltages, currents, transfer functions, and Bode Plots. We can also represent sinusoidal signals with complex numbers with phasors. It is critically important that we understand this chapter.
\end{abstract}  
\maketitle    
  
\begin{definition}
 A complex number  $z$ can be represented in the Cartesian coordinate system, as shown in Equation \ref{cartesian}, and  Polar coordinate system, as shown in Equation \ref{polar}.




\begin{eqnarray}
z= x + j y \label{cartesian} \\ 
z=|z| e^{j \Theta} \label{polar}
\end{eqnarray}

\end{definition}

In Equation \ref{cartesian}, a complex number z is represented in rectangular coordinate system, where $x$ is the real part, $y$ is the imaginary part, and $j=\sqrt{-1}$. 

 Equation \ref{polar}, shows a complex number z  in the polar coordinate system, where  $|z|$ is the magnitude, and $\Theta$ is the angle (aka phase) of the complex number.

The geometric interpretation of these two equations is shown in Figure \ref{cartpol}. The magnitude is the length of the triangle's hypothenuse, and the angle is the angle that the hypothenuse makes with the x-axis. 

 In Figure \ref{cartpol}, we  represent a complex number with a "position vector." Position vectors are vectors that start at the center of the coordinate system and end at any point in the coordinate system. We see that phasor is a vector that represents a complex number in a polar coordinate system. 



\begin{figure}[htbp]
\begin{center}
\includegraphics[scale=0.3]{../jpg/Complex_Numbersz12.jpg}
%\strut\psfig{figure=complexnumberz.ps,width=3cm} \\
\end{center}
\caption{Visual representation of a complex number z in rectangular $z=x+jy$ and polar coordinates $z=|z|e^{j \theta}$.}
\label{cartpol}
\end{figure}




You may be wondering why we represent the phase of a complex number in the polar coordinate system as  $e^{j \Theta}$ because in the circuits class, you used $ \angle \theta $. Great question. That brings us to Euler's formula that we will discuss in section Euler's Formula.

\section{Conversion between Cartesian and Polar coordinate systems}
  
  To find magnitude and angle when we know real and imaginary parts of a complex number, we use the Pythagorean Theorem to find the magnitude of the complex number as in Equation \ref{eq:polar_mag}, and use the definition of the tangent to find the angle as in Equation \ref{eq:polar_angle}.

\begin{eqnarray}
|z|=\sqrt{x^2+y^2} \label{eq:polar_mag}\\
\Theta = arctg \frac{y}{x} \label{eq:polar_angle}
\end{eqnarray}
 
 
 To find the real and imaginary part of a complex number when we know magnitude and phase, we use trigonometry. To find the real part of the complex number as in Equation \ref{eq:cart_real}, use the definition of cosine and sine to find the imaginary part of the complex number as in Equation \ref{eq:cart_imag}.


\begin{eqnarray}
\Re \{z\} =x= r \cos (\Theta) \label{eq:cart_real}\\
\Im \{z\} =y= r \sin (\Theta)  \label{eq:cart_imag}
\end{eqnarray}


\begin{question}
Explore the conversion of complex numbers between cartesian and polar coordinates.
\begin{center}  
\geogebra{b8hu8ztx}{800}{600}  
\end{center} 
\end{question}




\begin{example}
Find the magnitude and phase of complex numbers $z_1=j$ and $z_2=1$.


\begin{explanation}

Complex number $z_1=j$ is on the y-axis where y=1. By inspection, the magnitude 
     of $z_1$ is $|z|=1$, and the angle is $\theta=90^o$.



  \begin{image}
\begin{tikzpicture}
	\begin{axis}[
            xmin=-1.1,xmax=1.1,ymin=-1.1,ymax=1.1,
            axis lines=center,
            width=4in,
            xtick={-1,1},
            ytick={-1,1},
            clip=false,
            unit vector ratio*=1 1 1,
            xlabel=$x$, ylabel=$y$,
            every axis y label/.style={at=(current axis.above origin),anchor=south},
            every axis x label/.style={at=(current axis.right of origin),anchor=west},
          ]        
        %  \addplot [dashed, smooth, domain=(0:360)] ({cos(x)},{sin(x)}); %% unit circle

          \addplot [textColor] plot coordinates {(0,0) (0,1)}; %% 90 degrees

          %vertical red up line from 0,0 to 0,1
          \addplot [ultra thick,penColor2] plot coordinates {(0,0) (0,1)}; %% 90 degrees
          
         % draw small circle to designate that theta is 90 degrees        

          \addplot [textColor,smooth, domain=(0:90)] ({.15*cos(x)},{.15*sin(x)});
          % write theta next to the small circle
         
          \node at (axis cs:.11,.11) [anchor=west] {$\theta$};
          %write j next to the y=1
         
          \node at (axis cs:0.35,1) [anchor=east] {$z=j$};
          
        \end{axis}
\end{tikzpicture}
\end{image}
  

The complex plane is sketched below.
Complex number $z_1=1$ is on the x-axis where x=1. By inspection, the magnitude 
     of $z_1$ is $|z|=1$, and the angle is $\theta=0^o$.


  \begin{image}
\begin{tikzpicture}
	\begin{axis}[
            xmin=-1.1,xmax=1.1,ymin=-1.1,ymax=1.1,
            axis lines=center,
            width=4in,
            xtick={-1,1},
            ytick={-1,1},
            clip=false,
            unit vector ratio*=1 1 1,
            xlabel=$x$, ylabel=$y$,
            every axis y label/.style={at=(current axis.above origin),anchor=south},
            every axis x label/.style={at=(current axis.right of origin),anchor=west},
          ]        
        %  \addplot [dashed, smooth, domain=(0:360)] ({cos(x)},{sin(x)}); %% unit circle

          \addplot [textColor] plot coordinates {(0,0) (0,1)}; %% 90 degrees

          %horizontal red up line from 0,0 to 0,1
          \addplot [ultra thick,penColor2] plot coordinates {(0,0) (1,0)}; %% 0 degrees
          
         % draw small circle to designate that theta is 90 degrees        

      %    \addplot [textColor,smooth, domain=(0:90)] ({.15*cos(x)},{.15*sin(x)});
          % write theta next to the small circle
         
    %      \node at (axis cs:.11,.11) [anchor=west] {$\theta$};
          %write j next to the y=1
         
          \node at (axis cs:1,0) [anchor=south] {$z=1$};
          
        \end{axis}
\end{tikzpicture}
\end{image}


\end{explanation}


\end{example}


\begin{question}
Calculate magnitude and phase of 
complex number $z=-j$ 
\begin{multipleChoice}  
\choice{$|z|= 1, \theta={180}$}
\choice[correct]{$|z|= 1, \theta={-90}$}   
\choice{$|z|= -1, \theta={180}$}
\choice{$|z|= -1,  \theta={-90}$}
\end{multipleChoice}
\end{question}
  
  
  
  
  
  
  
  
  
  
  
  
  
  
  
  
  
  
  
  
  
  
  
  Write magnitude and phase of complex number $z=-1$
\begin{question}  
         $ -1 =  \answer{e^{j180}}$  
    \end{question} 
    
    
    
    
    
    
    
    
    
    
    
    
    
    
    
    
    
    
    
\end{document} 
