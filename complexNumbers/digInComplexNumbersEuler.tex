\documentclass{ximera}  

\input{../preamble.tex}

 
\title{Euler's Formula} 
\author{Milica Markovic} 
\outcome{Explain how Euler's formula relates sinusoidal signals and complex numbers.}

\begin{document}  
\begin{abstract}  
The purpose of this section is to relate sinusoidal signals and complex numbers using Euler's formula.
\end{abstract}  
\maketitle    
  



You may be wondering why we represent the phase of a complex number in the polar coordinate system as  $r e^{j ^Theta} $ because, in the circuits class, we used $\angle \theta$. Great question. That brings us to Euler's formula.

  
  

 Euler's formula relates Cartesian and Polar coordinates for complex numbers.

\begin{eqnarray}
 e^{j \Theta} =  \cos \Theta + j  \sin \Theta
\end{eqnarray}

Geometric interpretation of the Euler's formula is shown below. $z=r ( \cos{\theta} + j \sin{\theta})$, where $r \cos{\theta}=x$ and $r \sin{\theta}=y$. Euler's formula shows that number z given in Cartesian coordinates as $x+jy$ can be represented in Polar Coordinates as   $e^{j \Theta}$. You have likely seen this proof in your Calculus class.  {\bf TIP: Your calculator may not know what $e^{j\theta}$ is. Check how to convert between polar and cartesian coordinates on your calculatorbefore the test.}


  \begin{image}
\begin{tikzpicture}
	\begin{axis}[
            xmin=-1.1,xmax=1.1,ymin=-1.1,ymax=1.1,
            axis lines=center,
            width=4in,
            xtick={-1,1},
            ytick={-1,1},
            clip=false,
            unit vector ratio*=1 1 1,
            xlabel=$x$, ylabel=$y$,
            every axis y label/.style={at=(current axis.above origin),anchor=south},
            every axis x label/.style={at=(current axis.right of origin),anchor=west},
          ]        
          \addplot [dashed, smooth, domain=(0:360)] ({cos(x)},{sin(x)}); %% unit circle

          \addplot [textColor] plot coordinates {(0,0) (.766,.643)}; %% 40 degrees

          \addplot [ultra thick,penColor] plot coordinates {(.766,0) (.766,.643)}; %% 40 degrees
          \addplot [ultra thick,penColor2] plot coordinates {(0,0) (.766,0)}; %% 40 degrees
          
          %\addplot [ultra thick,penColor3] plot coordinates {(1,0) (1,.839)}; %% 40 degrees          

          \addplot [textColor,smooth, domain=(0:40)] ({.15*cos(x)},{.15*sin(x)});
          %\addplot [very thick,penColor] plot coordinates {(0,0) (.766,.643)}; %% sector
          %\addplot [very thick,penColor] plot coordinates {(0,0) (1,0)}; %% sector
          %\addplot [very thick, penColor, smooth, domain=(0:40)] ({cos(x)},{sin(x)}); %% sector
          \node at (axis cs:.15,.07) [anchor=west] {$\theta$};
          \node[penColor, rotate=-90] at (axis cs:.84,.322) {$\sin(\theta)$};
          \node[penColor2] at (axis cs:.383,0) [anchor=north] {$\cos(\theta)$};
          %\node[penColor3, rotate=-90] at (axis cs:1.06,.322) {$\tan(\theta)$};
        \end{axis}
\end{tikzpicture}
\end{image}


Voltages and currents that we represent have different amplitudes that sometimes change with time, as we will see in this course in the section on transmission lines. We can multiply Euler's formula with a constant, and get the general form that includes the amplitude of signals $e^{-\sigma t}$.


\begin{eqnarray}
 e^{-\sigma t}e^{j \Theta} =  e^{-\sigma t}\cos \Theta + j e^{-\sigma t} \sin \Theta \nonumber \\
 e^{(-\sigma + j \omega) t} =  e^{-\sigma t}\cos \Theta + j e^{-\sigma t} \sin \Theta
\end{eqnarray}



If we further replace the angle $\Theta$ with $\omega t$, we can use Euler's formula to represent sinusoidal signals that vary with time $\cos(\omega t)$ and $\sin (\omega t)$. 


Observe the simulation below. On the left side is the complex number $e^{-\sigma +j \omega t} = e^{-\sigma t} \cos(\omega t) + j e^{-\sigma t} \sin (\omega t)$ whose value on the real axis represents the real part of the complex number $e^{-\sigma t} \cos(\omega t)$ , and imaginary axis represents the imaginary part of the complex number $e^{-\sigma t} \sin (\omega t)$. We see the real and imaginary parts of this complex number on the right side, where cosine and sine functions in time-domain. We use this relationship in circuit analysis. The complex number is called a "phasor". Phasors simplify circuit analysis as we will see in the section on Phasors.

  
 \begin{center}  
\geogebra{tanbujyz}{1500}{1000}  
\end{center} 

\end{document} 
