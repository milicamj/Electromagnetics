\documentclass{ximera}  


\input{../preamble.tex}



 
\title{Power} 
\author{Milica Markovic} 
\outcome{Design impedance matching circuits.}
\begin{document}  
\begin{abstract}  

\end{abstract}  
\maketitle    



The current and voltage phasor transformation is defined as

\begin{eqnarray}
v(t)=\Re\{|V|e^{j\theta_v} e^{j \omega t}\} \\
i(t)=\Re\{|I|e^{j\theta_i} e^{j \omega t}\}
\end{eqnarray}

Where $|V|e^{j\theta_v}$ and $|I|e^{j\theta_i}$ are phasors of voltage and current, and usually denoted with a tilde $\tilde{}$ over capital letters  $\tilde{V}$, $\tilde{I}$.


\begin{eqnarray}
v(t)=\Re\{\tilde{V} e^{j \omega t}\} \\
i(t)=\Re\{\tilde{I} e^{j \omega t}\}
\end{eqnarray}


The real part of a complex number can also be found as $\Re\{z\}=\frac{1}{2}(z +z^*)$, so the above two equations can be re-written as

\begin{eqnarray}
v(t)=\frac{1}{2} ( \tilde{V}e^{j \omega t} + \tilde{V}^*e^{-j \omega t} ) \label{phasorV}\\
i(t)=\frac{1}{2} ( \tilde{I}e^{j \omega t} + \tilde{I}^*e^{-j \omega t} ) \label{phasorI}\\
\end{eqnarray}

The power is defined as a product of voltage in current.

\begin{equation}
p(t) = v(t) i(t)
\end{equation}

If we replace voltage and current in the time domain with Eq \ref{phasorV} and \ref{phasorI} we get



\begin{eqnarray}
p(t) =\frac{1}{4} ( \tilde{V}e^{j \omega t} + \tilde{V}^*e^{-j \omega t} ) ( \tilde{I}e^{j \omega t} + \tilde{I}^*e^{-j \omega t} ) \\
p(t)=\frac{1}{4} (\tilde{V}\tilde{I}^*+ (\tilde{V}\tilde{I}^*)^*+ \tilde{V}\tilde{I}^* e^{2j \omega t}+ (\tilde{V}\tilde{I}^* e^{2j \omega t})^*)
\end{eqnarray}

We can again apply equation $\Re\{z\}=\frac{1}{2}(z +z^*)$ to simplify the above equation to

\begin{eqnarray}
p(t)=\frac{1}{2} (\Re\{\tilde{V}\tilde{I}^*\}+ \Re\{\tilde{V}\tilde{I} e^{2j \omega t }\})
\end{eqnarray}

This can also be re-written as

\begin{eqnarray}
p(t)=\frac{1}{2} \Re\{ |V| e^{j\theta_v} |I| e^{-j\theta_i} \}+ \frac{1}{2} \Re\{|V| e^{j\theta_v} |I| e^{j\theta_i} e^{2j \omega t }\} \\
p(t)=\frac{1}{2} \Re\{ |V| |I| e^{j(\theta_v-\theta_i)} \}+ \frac{1}{2} \Re\{|V| |I|  e^{j(\theta_v+\theta_i)}  e^{2j \omega t }\}
\end{eqnarray}


p(t) above is instantaneous power, $S=\tilde{V}\tilde{I}^*=|V| |I| e^{j(\theta_v-\theta_i)}$ is complex power. Complex power has real and reactive parts S=P+jQ. The first part of the equation represents the average real power P delivered to the load $P=\frac{1}{2}\Re\{\tilde{V}\tilde{I}^*\}$, and the second part represents the fluxtuating power. We are usually interested in the average real power P delivered to the load.

To find the real power delivered to the load, one would take the real part of the complex power. If we know that the impedance of the load is $Z=R+jX$, the voltage is $ \tilde{V} = Z \tilde{I}$ and we remember that $\tilde{I} \tilde{I}^* = |I|^2$ then the real power is

\begin{eqnarray}
P=\frac{1}{2}\Re\{ \tilde{V} \tilde{I}^*  \} \\
P=\frac{1}{2}\Re\{ (R+jX) \tilde{I} \tilde{I}^*  \} \\
P=\frac{1}{2}\Re\{ (R+jX) |I|^2  \} \\
P=\frac{1}{2}|I|^2 \Re\{ (R+jX)   \} \\
P=\frac{1}{2}R |I|^2 
\end{eqnarray}




\begin{example}
A transmitter operated at 20MHz, Vg=100V with $50 \Omega$ internal impedance is connected to an antenna load through 6.33m of the line. The line is a lossless $50 \Omega$, $\beta=0.595rad/m$. The antenna impedance at 20MHz measures $Z_L=36+j20 \Omega$. 
\begin{enumerate}
\item What is the electrical length of the line? (answer: length=0.6$\lambda$)
\item How much power is delivered to the line? Hint: Find the input impedance, then find the input power as $P_{ave,in}=\frac{1}{2}R_{in} |I_{in}|^2$
\item What is the time-average power absorbed by $Z_L$. $P_{L}=\frac{1}{2} R_L |I_{L}|^2$
\item If now we match load impedance $Z_l$ to 50 Ohm line, what is the input impedance of the line, and how much average power is delivered to the line and load?
\end{enumerate}

\end{example}

\end{document} 
