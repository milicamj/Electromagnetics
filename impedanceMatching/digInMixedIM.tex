\documentclass{ximera}  


\input{../preamble.tex}



 
\title{Mixed Impedance Matching} 
\author{Milica Markovic} 
\outcome{ Design a mixed impedance matching network for any load impedance and discuss pros and cons of various designs.}
\begin{document}  
\begin{abstract}  

\end{abstract}  
\maketitle    


The real part of the load impedance is rarely equal to the transmission-line impedance. In most cases, we have to transform the real part of the impedance as well. 


For example, load impedance $Z_L=25+j 50 \Omega$ represents a series connection of a $25\Omega$ resistor and a $7.96$\,nH inductor at 1\,GHz. To match $Z_L=25+j 50 \Omega$ impedance to the transmission-line impedance $Z_0=50 \Omega$, we first normalize the load impedance to transmission-line impedance.


\begin{equation}
\bar{Z}_L=\frac{Z_L}{Z_0}=0.5+j1
\end{equation}

This impedance is shown in Figure \ref{fig:PointSC}. 

\begin{figure}[htbp]
\begin{center}
\includegraphics[scale=0.5]{../jpg/MixedMatch2-01.jpg}
\end{center}
\caption{Load impedance $Z_L=0.5+j1$ on Smith Chart.}
\label{fig:PointSC}
\end{figure}


\subsection*{SWR circle}
Then, we identify an SWR circle that this impedance is on, as shown in Figure \ref{fig:SWRfor25Ohm}. The point where the SWR circle intersects the green circle is of interest because the real part of the input admittance is equal to one $Y_1=1$. This second point, where $Y=1+j1.6$, will give us the length of the line that we have to add to the load impedance. 


\begin{example}
Explain how the position of the load impedance on the Smith Chart changes the SWR circle.
\begin{explanation}
To see how SWR circle changes depending on where the load impedance is use the following simulation. Click on point A, to change its position. Observe how SWR circle changes.
\begin{center}  
\geogebra{ugy2wcxc}{800}{600}  
\end{center} 
\end{explanation}

\end{example}

\subsection{Length of the line that will transform the real part of load impedance}

To find the length of the line that will transform the real part of $z_L$ to $y_{1}=1$, we identify the position of the load impedance $Z_L=0.5+j1$, and the input impedance $Z_1=0.3-j0.45$ at the {\it Wavelengths Towards Generator} (WTG) scale. The reason we picked impedance $Z_1=0.3-j0.45$ is because the real part of the admittance $y_1=1/z_1$ is equal to one $\Re(y_{1})=1$. Load impedance  $Z_L$ is at $0.135 \lambda$, and the input impedance $Z_1$ is at $0.425 \lambda$. The difference between these two positions gives us the length of the line $0.29 \lambda$. In electrical degrees, this length is approximately $105^0$. The input admittance to the line is now $Y_L=1+1.6$.

\begin{figure}[htbp]
\begin{center}
\includegraphics[scale=1]{../jpg/MixedMatch4-01.jpg}
\end{center}
\caption{SWR circle for impedance $Z_L=0.5+j1$.  }
\label{fig:SWRfor25Ohm}
\end{figure}

\subsection{Adding a lumped-element to remove the susceptance}


The final step is to add a susceptance that will remove the imaginary part of the input admittance $Y_1=1+j 1.6$. We see that to get the final admittance of $Y_M=1$, numerically, we have to add an admittance of $Y_{add}=-j1.6$. This represents an inductance. Since we are adding the two admittances $Y_1+Y_{add}$, they have to be in parallel (as we know that when elements are in parallel, we add their admittances).


\begin{figure}[htbp]
\begin{center}
\includegraphics[scale=1]{../jpg/MixedMatch5-01.jpg}
\end{center}
\caption{The result of impedance matching.}
\label{impmatchgen}
\end{figure}
\newpage

\subsection{Other possible solutions}

Graphically, there are several different mixed or transmission-line impedance matching circuits that we can make for a specific impedance. For example, for impedance $Z_L=25+j50 \Omega$, $z_L=0.5+j1$, there are four different circuits that we can make, as shown in Figure \ref{fig:MixedVariety}. In this paragraph, we used the green path on the Smith Chart, with intermediate admittance $Y_2$. 


\begin{figure}[htbp]
\begin{center}
\includegraphics[scale=1]{../jpg/MixedVariety-01-01.jpg}
\end{center}
\caption{A variety of possible impedance matching circuits for impedance $Z_L=25+j50 \Omega$.}
\label{fig:MixedVariety}
\end{figure}




\end{document} 



