\documentclass{ximera}  

\input{../preamble.tex}

\title{Basic Parameters of Sinusoidal Signals}  
\author{Milica Markovic}
\outcome{Recognize the basic properties of sinusoidal functions and relate them to circuit analysis. Describe basic properties of sinusoidal functions. Compare and contrast phase with time-delay.}
\begin{document}  
\begin{abstract}  
Review of Sinusoidal Signals
\end{abstract}  
\maketitle

\section{Sinusoidal signal parameters}

Sinusoidal signals are important because in electrical engineering we use them to analyze and test circuit performance. All periodic signals can be represented with sinusoidal signals of different amplitudes and phases using the Fourier series. 

A typical sinusoidal signal is shown in Figure \ref{sinusoid}. On the y-axis is the instantaneous value of the sinusoidal voltage and on the x-axis is time. Instantaneous values of voltage change from -1V to 1V with time. Sinusoidal signals can be characterized by the following parameters: peak amplitude, peak-to-peak, average, RMS, period, time-delay, and phase. Peak amplitude, peak-to-peak, average and RMS values, are read on the y-axis in Figure \ref{sinusoid}, whereas period, time delay and phase are read on the x-axis.

\begin{figure}[htbp]
\begin{center}
\includegraphics[scale=0.4]{jpg/sinusoid.jpg}
\caption{Vocabulary used in describing sinusoidal signals.}
\label{sinusoid}
\end{center}
\end{figure} 


\subsection{Parameters that are read on y-axis.}

\begin{definition}
Peak amplitude is measured on the y-axis as the length from the average value of the signal (in this case zero) to the maximum value of the signal (in this case 1). For signal shown in Figure \ref{sinusoid}, the peak  amplitude has a constant value of $V_p=1$. 
\end{definition}


\begin{definition}
Instantaneous value of the sinusoidal signal varies from -1 to 1V, and depends on where on the x-axis are we observing the instantaneous value. 

Compared to the instantaneous value, the peak amplitude is allways constant and it does not vary with time. 
\end{definition}

\begin{definition}
Peak-to-peak is measured from the minimum value of the function (in this case -1) to the maximum value of the function (in this case 1).  For signal shown in Figure \ref{sinusoid}, peak-to-peak voltage has a constant value of  $V_{pp}=2$.
\end{definition}

\begin{definition}
RMS or root-mean-square is defined as $v_{rms}=\frac{1}{T} \sqrt{\int_0^T v(t)^2 dt}$. For signal shown in Figure \ref{sinusoid}, and other sinusoidal signals of this form,  $v_{rms}=\frac{V_p}{\sqrt{2}}=\frac{1}{\sqrt{2}}=0.707$. Root mean square value is important because it represents the equivalent amount of DC power.
\end{definition}


\begin{definition}
Average value $v_{ave1}=\frac{1}{T} \int_0^T v(t) dt$. For the signal shown in Figure \ref{sinusoid}, the average value is $V_{ave1}=0$ because the function has the same area under the function in the positive and negative cycle. 
\end{definition}

\subsection{Parameters that are read on the x-axis.}

\begin{definition}
Sinusoidal signals can be represented as a function of time, Figure \ref{sin}, or a function of angle, Figure \ref{sinPh}. Take a few minutes to see how the graphs are the same and how are they different.



\begin{figure}[htpb]
\includegraphics[scale=0.4]{jpg/cpef1.jpg}
\caption{$sin ( \omega t)$ as a function of time.} \label{sin}
\end{figure}




\begin{figure}[htpb]
\includegraphics[scale=0.4]{jpg/cpef3.jpg}
\caption{Sinusoidal signal as a function of angle $\omega t$.}
\label{sinPh}
\end{figure}

\end{definition}


\begin{definition}
Period, T, is measured on the x-axis as the length of one full cycle of the sinusoidal signal. For signal shown in Figure \ref{sinusoid}, this value is $period=T$. 
\end{definition}

\begin{definition}
Frequency, f,  is defined as a reciprocal value of the period T, $f=\frac{1}{T}$. It represents how fast is the signal changing in time.  In Figure \ref{sinF1F2}, sinusoidal signals of two different frequencies f are given. 

\begin{figure}[htbp]
\includegraphics[scale=0.4]{jpg/cpef6.jpg}
\caption{Sinusoidal signals of different frequencies $sin ( \omega t)$}
\label{sinF1F2}
\end{figure}

\end{definition}

\begin{definition}
Time delay and phase represent the lag (or lead) of one function with respect to another in time domain and frequency domain. For example, in Figure \ref{sinusoid}, function $ \cos(\omega t - 90^o)$ is time-delayed for $\tau = \frac{T}{4}$ with respect to $\cos (\omega t)$. To find the time delay for a sinusoidal signal from its phase, we look at the way to represent the phase $90^o$ in terms of the product of frequency and time. Since in the sinusoidal signal expression $\cos (\omega t + \Theta)$  phase $\Theta$ is added to $\omega t$ term, the phase has the same units as $\omega t$, and can be represented as the product of $\omega \tau = \theta$, $\tau = \frac{\theta}{\omega}$, where $\tau$ represents the time delay.
\end{definition}









\begin{figure} [htbp]
\includegraphics[scale=0.4]{jpg/cpef4.jpg}
\caption{Sinusoidal signal as a function of angle $\omega t$ with a phase shift of $-\pi/4$}
\label{sinMinus45Ph}
\end{figure}



\begin{question}  
Calculate the time-delay in nanoseconds that you would observe on an oscilloscope if the frequency of the signal is f=0.159\,GHz and the phase shift of the signal is $\theta=10^o$. \\
$ \frac{\theta}{2*\pi*f} = \answer{10}$  ns
\end{question} 

\begin{example}
Observe the three signals below, and change their amplitude and phase. Explain qualitatively how are the signals changing when we move the sliders?
\begin{center}  
\geogebra{w8r4epyh}{800}{600}  
\end{center} 
\end{example}





\end{document}



                                                                   



                                       




                                    















