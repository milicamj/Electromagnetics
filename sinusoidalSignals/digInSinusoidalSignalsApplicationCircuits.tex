\documentclass{ximera}  

\input{../preamble.tex}

\title{eLi the iCe Man is CIVIL}  
\author{Milica Markovic}
\outcome{Recognize the basic properties of sinusoidal functions and relate them to circuit analysis.}
\begin{document}  
\begin{abstract}  
Review of Sinusoidal Signals
\end{abstract}  
\maketitle

In a resistor, the current and voltage maximums occur simultaneously, but in capacitors and inductors, maximums of current and voltage occur at different times.

``eLi the iCe man" and ``CIVIL" are mnemonics to help us remember how the current and voltage lag or lead in inductors and capacitors. 

On an inductor , maximum voltage occurs before the current maximum (sometimes voltage is labeled as "e", eLi, VIL). In an inductor, we say that the voltage leads the current, or the current lags the voltage. 

In a capacitor, the current maximum occurs before the voltage maximum (iCe, CIV). In a capacitor, we say that the current leads the voltage, and voltage lags the current. By observing voltage and current maximums, we can tell if a capacitor or inductor produced the voltages and currents displayed. 

\begin{example}
Observe just the sinusoidal voltages and currents in the series RLC circuit below. Can you apply CIVIL or eLi the iCe man? You do not have to worry about the phasor vector diagrams; we'll talk about that later.

\begin{center}  
\geogebra{dgftaeya}{1000}{800}  
\end{center} 
\end{example}
\end{document}



                                                                   



                                       




                                    















