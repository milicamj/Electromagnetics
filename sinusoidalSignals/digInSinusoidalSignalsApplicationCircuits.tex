\documentclass{ximera}  

\input{../preamble.tex}

\title{eLi the iCe Man is CIVIL}  
\author{Milica Markovic}
\outcome{Recognize the basic properties of sinusoidal functions and relate them to circuit analysis.}
\begin{document}  
\begin{abstract}  
Review of Sinusoidal Signals
\end{abstract}  
\maketitle

In a resistor, the current and voltage maximums occur at the same time, but in capacitors and inductors, maximums of current and voltage occur at different times.

``eLi the iCe man" and ``CIVIL" are mnemonics to help us remember how the current and voltage lag or lead in inductors and capacitors. In an inductor voltage (sometimes labeled as e) maximum occurs before the current (eLi, VIL), and in a capacitor, current maximum occurs before the voltage maximum (iCe, CIV). This means that in an inductor, voltage leads the current,  current lags the voltage, and in the capacitor, current lags the voltage, and voltage leads the current. By observing voltage and current, we can tell if the mistery impedance is a capacitor or inductor. 

\begin{example}
Observe just the sinusoidal voltages and currents in the series RLC circuit below. Can you apply CIVIL or eLi the iCe man? You don't have to worry about the phasor vector diagrams, we'll talk about that later.

\begin{center}  
\geogebra{dgftaeya}{1000}{800}  
\end{center} 
\end{example}
\end{document}



                                                                   



                                       




                                    















