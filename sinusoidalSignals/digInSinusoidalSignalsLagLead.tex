\documentclass{ximera}  

\input{../preamble.tex}

\title{Leading and Lagging Signals}  
\author{Milica Markovic}
\outcome{Recognize leading and lagging signals. Explain why is a signal leading or lagging.}
\begin{document}  
\begin{abstract}  
Review of Sinusoidal Signals
\end{abstract}  
\maketitle


\begin{definition}
How do we recognize lagging and leading on a graph? 

In Figure \ref{timedelaysig} we observe two step functions, V(t) and V(t-T). Function V(t) step occurs at t=0, and V(t-T) step occurs at t=T. The function V(t-T) is shifted to the right, the step occurs later, at t=T, and is, therefore, lagging function V(t). 


Similarly, if the step function is V(t+T), the function v(t) is shifted to the left. The step occurs earlier at t=-T, and therefore V(t+T) is leading V(t).


\begin{figure}[htbp]
\begin{center}
\includegraphics[scale=0.5]{jpg/timedelayedsignal.jpg}  
%\strut\psfig{figure=generaltransmissionlinecircuit.ps,width=3cm}
\end{center}
\caption{Voltage as a function of time at the generator side (top) and the load side (bottom) of a transmission line, if the switch closes at t=0 the voltage arrives at t=l/c=T at the load. These graphs can be obtained by observing the voltage on an oscilloscope at the load and at the generator side.}
\label{timedelaysig} 
 \end{figure}
\end{definition}



\begin{example}

What if we have a sinusoidal signal? We will observe a specific point on the signal, such as the maximum value, and determine if it shifted left or right on the graph.

When the phase of a signal is positive as in Figure \ref{sinPlus45Ph} $ \sin (\omega t + 45^o)$, we say that the signal is leading with respect to the signal $v(t)= \sin (\omega t)$, because it is shifted to the left for $45^o$($pi/4$). The maximum of the function now occurs at t=-T, or $\omega t = -45^o$, and we can write the new function as the original sinusoidal function V(t) shifted left for a time T,  V(t+T). The phase of the signal is $45^o$, and the time-delay is T. 


\begin{figure}[htbp]
\includegraphics[scale=0.4]{jpg/cpef5.jpg}
\caption{Sinusoidal signal as a function of angle $\omega t$ with a phase shift of $+\pi/4$}
\label{sinPlus45Ph}
\end{figure}

\end{example}


\begin{example}
\item When the phase of a signal is negative as in Figure \ref{sinMinus45Ph}, \ref{sinMinus45T},  $ \sin (\omega t - 45^o)$, we say that the signal is lagging with respect to the signal $ \sin (\omega t)$, because it is shifted to the right for $45^o$ ($pi/4$), or $\tau=-\frac{pi/4}{\omega} $. The lagging function's peak occurs later in time, and therefore it is lagging. The phase of the signal is $-45^o$.


\begin{figure}[htbp]
\begin{center}
\includegraphics[scale=0.4]{jpg/cpef2.jpg}
\caption{ Sinusoidal signal shifted for time delay $-\frac{\pi/4}{\omega}$}
\label{sinMinus45T}
\end{center}
\end{figure}


\begin{figure}[htbp]
\begin{center}
\includegraphics[scale=0.4]{jpg/cpef4.jpg}
\caption{ Sinusoidal signal with phase shift $-\pi/4$}
\label{sinMinus45Ph}
\end{center}
\end{figure}

\end{example}

\begin{question}
Sinusoidal signal $v_1=\cos(\omega t - 25^o)$ is given. Compared to $v=\cos(\omega t)$, signal $v_1$
\begin{multipleChoice}  
\choice{Leads signal $v$}
\choice[correct]{Lags signal $v$}   
\end{multipleChoice}
\end{question}



\begin{question}  

Observe three signals  in Figure below


\begin{image}
\begin{tikzpicture}
    \begin{axis}[
            xmin=-6.75,xmax=6.75,ymin=-1.5,ymax=1.5,
            axis lines=center,
            xtick={-6.28, -4.71, -3.14, -1.57, 0, 1.57, 3.142, 4.71, 6.28},
            xticklabels={$-2\pi$,$-3\pi/2$,$-\pi$, $-\pi/2$, $0$, $\pi/2$, $\pi$, $3\pi/2$, $2\pi$},
            ytick={-1,1},
            %ticks=none,
            width=6in,
            height=3in,
            unit vector ratio*=1 1 1,
            xlabel=$\theta$, ylabel=$x$,
            every axis y label/.style={at=(current axis.above origin),anchor=south},
            every axis x label/.style={at=(current axis.right of origin),anchor=west},
          ]        
          \addplot [very thick, penColor, samples=100,smooth, domain=(-6.75:6.75)] {cos(deg(x))};
        \addplot [very thick, penColor4, samples=100,smooth, domain=(-6.75:6.75)] {cos(deg(x)+90)};
         \addplot [very thick, penColor2, samples=100,smooth, domain=(-6.75:6.75)] {cos(deg(x)-90)};
                   
          \node at (axis cs:-3,-1.2) [penColor] {$\cos(\theta)$};
          \node at (axis cs:-1.72, 1.2) [penColor4] {$\cos(\theta + \pi/4)$};
        \node at (axis cs:3, 1.2) [penColor2] {$\cos(\theta - \pi/4)$};
        \end{axis}
\end{tikzpicture}
%% \caption{The function $\cos(\theta)$ takes on all values between $-1$
%%   and $1$ exactly once on the interval $[0,\pi]$. If we restrict
%%   $\cos(\theta)$ to this interval, then this restricted function has
%%   an inverse.}
%% \label{figure:cos-restricted}
%% \end{figure*}
\end{image}


Which of the following functions leads $\cos(\omega t)$?  
\begin{multipleChoice}  
\choice[correct]{The green signal.}  
\choice{The red signal.}  
\choice{The blue signal.}  
\end{multipleChoice}  
\end{question}


\end{document}



                                                                   



                                       




                                    















