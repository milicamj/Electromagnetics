\documentclass{ximera}  


\input{../preamble.tex}



 
\title{Kirchoff's Laws} 
\author{Milica Markovic} 
\outcome{Apply phasor transformation to Kirchoff's Laws.}
\begin{document}  
\begin{abstract}  
Phasors are essential tool in circuit analysis.
\end{abstract}  
\maketitle    



We will see how the phasor transformation is applied to an RC circuit shown in Figure 2\ref{RCcirc}.  To solve this circuit in the time domain we apply Kirchoff's voltage law as shown in Equation \ref{eq-1} -\ref{eq0}.


The circuit in Figure \ref{RCcirc} is a simple RC circuit. KVL equation in the time domain is given in Equation \ref{eq-1}.

\begin{eqnarray}
        v_s(t)=v_R(t)+v_C(t)                \label{eq-1}  \\
v_s(t) = R i + \frac{1}{C} \int i(t) dt \label{eq0}
\end{eqnarray} 


As we discussed in the previous section, we will be using principle of superposition, and add another generator to the circuit, as shown in Figure \ref{RCcirc}.

\begin{figure}[htbp]
\begin{center}
\includegraphics[scale=0.5]{../jpg/RCcircuitPhasorSup.jpg}
%\strut\psfig{figure=complexnumberz.ps,width=3cm} \\
\end{center}
\caption{Using superposition to find phasors of voltages and currents in an RC circuit.}
\label{RCcirc}
\end{figure}

The generator we originally had in the circuit is now just the real part of the phasor expresion shown in Equation \ref{eq2}.


\begin{eqnarray}
v_s(t)=  V \cos (\omega t + \Theta_V)=\Re\{ A cos (\omega t + \Theta ) + j A sin (\omega t + \Theta)\}= \nonumber \\ 
= \Re\{A e^{j(\omega t + \Theta)}\}=\Re\{A e^{j \Theta} e^{j \omega t}\} \label{eq2}
\end{eqnarray}
We can now use the analysis from the previous section to  replace the time-domain quantities in equation \ref{eq0} with
these newly developed expressions.



\begin{eqnarray}
v_s(t)=v_R(t) + v_C(t)  \\
Re\{A e^{j \Theta} e^{j \omega t}\}=    \Re\{R \times I e^{j \Theta_I} e^{j \omega t}\}   +  \Re\{   \frac{1}{j \omega C}  I e^{j \Theta_I}  e^{j \omega t}  \}
\end{eqnarray}

A common term in the previous equation is $ e^{j\omega t}$, and we can now drop $Re$, as long as we later remember to take only the real part of the expresion for the phasor of voltage and current to get the time domain expression. We can now
write the equation as




\begin{eqnarray}
 V_S  e^{j \Theta_{V_S}}  =R \times I e^{j \Theta_I}  +    \frac{1}{j \omega C}  I e^{j \Theta_I}  \\
V_S(j \omega)   = R I(j \omega)    + \frac{I(j \omega )}{j \omega C} 
\end{eqnarray}



Since this is a linear equation, we can easily solve it:


\begin{eqnarray}
I (j \omega) = \frac{V_S(j \omega)}{ R    + \frac{1}{j \omega C} } \label{pheq}
\end{eqnarray} 



Let's say that the values  are given for R, C,  $\omega$  and $V_S$ such that the phasor of the current is  $I=3 e^{j 45^o}$. To obtain the signal in the time domain,  we   multiply the phasor $I$ with the $ e^{j\omega t}$ term, and  then we find
the real part of the expression to obtain its current in the time domain, as shown in Figure \ref{phtotd}


\begin{eqnarray}
i(t) = \Re\{ 3 e^{j 45^o}  e^{j\omega t} \} =  Re\{3 e^{j \omega t + 45^o}  \} = \nonumber \\ = \Re \{ 3 \cos (\omega t + 45^o ) + j 3 \sin (\omega t + 45^o \} = 3  \cos (\omega t + 45^o ) \label{phtotd}
\end{eqnarray}

In case we have an inductor in the circuit, the voltage on an inductor can be derived as shown in Figure \ref{eq77}. 


\begin{eqnarray}
v_L(t) = L \frac{\partial{ i(t)}}{\partial t}  =L  \frac{ Re\{    I e^{j \Theta_I} e^{j \omega t}\}}{\partial t}  dt  = \Re\{  LI e^{j \Theta_I} \frac{ \partial e^{j \omega t}}{\partial t} \} =  \nonumber \\ = \Re\{  LI e^{j  \Theta_I}  j  \omega e^{j \omega t} \} =  \Re \{  j  \omega       LI e^{j  \Theta_I}    e^{j \omega t}   \}      \label{eq77} 
\end{eqnarray}


\begin{figure}[htbp]
\begin{center}
\includegraphics[scale=0.3]{../jpg/RLnew.jpg}
%\strut\psfig{figure=complexnumberz.ps,width=3cm} \\
\end{center}
\caption{Complex number z in rectangular and polar coordinates.}
\label{RLcirc}
\end{figure}







\begin{table}
\centering
\begin{tabular}{|c|c|c|c|c|} \hline
cicruit element & impedance & low frequencies $f \to 0$& high frequencies $f \to \inf$   \\  \hline  
 capacitor     & $\frac{1}{j \omega C}$    & $\infty$ & $0$    \\  \hline       
 inductor & $j \omega L$              &    $0$   &       $\infty $             \\ \hline
\end{tabular}
\caption{Impedance of the capacitor and inductors and their equivalent impedances at high and low frequencies.}
\end{table}


\end{document} 
