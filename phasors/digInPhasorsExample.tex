\documentclass{ximera}  


\input{../preamble.tex}



 
\title{Example of circuit analysis with phasors} 
\author{Milica Markovic} 
\outcome{Apply phasor transformation to a time-domain equation to obtain frequency-domain equation.}
\begin{document}  
\begin{abstract}  
Phasors are essential tool in circuit analysis.
\end{abstract}  
\maketitle    


It is customary in Electrical Engineering to use  $\cos( \omega t)$ for our time-domain signal, and we call these phasors "cosine reference phasors".  If the generator in a circuit is given in terms of $\sin (\omega t)$, the sin function has to be converted to a cosine. To do that, subtract $90^o$ from the phase of the sinusoid, because $\sin( \omega t) = \cos(\omega t - 90^o)$.

\begin{example}
Using Vectors to Represent Phasors in an example


 Calculate on paper the magnitude  and phase of the current and voltages in a series RC circuit shown in Figure \ref{f112} (a) if the circuit is driven with a frequency of 1\,GHz (phase is zero) and R=1k$\Omega $, C= ${\frac{1}{2\pi }10}^{-12}$F. 
 
 
 Answer the following questions:

\begin{enumerate}
\item  What are the magnitude, phase and time delay of the source voltage?

\item  What are the magnitude, phase and time delay of the voltage across the resistor?

\item  What are the magnitude, phase and time delay of the voltage across the capacitor?

\item  The simulated voltage magnitude across the resistor is about 0.5V and the simulated voltage across the capacitor is about 0.85V.  If we use KVL $ 0.85V+0.5V \neq 1V$. Why? Look at Figure \ref{f112} to help you with answer this question.

\end{enumerate}
\begin{explanation} 
 
 
  When you calculate the magnitudes and phases of the voltages in the RC circuit, draw three points, to represent magnitude and phases of all three vectors, as in Figure \ref{PP}. The three points denote the three position vectors as shown in Figure \ref{f112}. Vector addition must be used to add the voltages on the resistor and capacitor to obtain the voltage of the generator. 






\begin{figure}[htbp]
%\vspace*{-0.5cm}
\begin{center}
\includegraphics[scale=0.2]{../jpg/RCcircADS.jpg}
\end{center}
\caption{\label{RCcircADS} RC circuit in ADS.}
\end{figure}


\begin{figure}[htbp]
%\vspace*{-0.5cm}
\begin{center}
\includegraphics[scale=0.3]{../jpg/voltagesinRCcircRadADS}
\end{center}
\caption{\label{SSangle} Sinusoidal signal as a function of angle $\omega t$}
\end{figure}


\begin{figure}[htbp]
%\vspace*{-0.5cm}
\begin{center}
\includegraphics[scale=0.2]{../jpg/RCvoltPolarPlotADS.jpg}
\end{center}
\caption{\label{PP} Points represent polar plot of complex voltages in RC circuit.}
\end{figure}


\begin{figure}[htbp]
%\vspace*{-0.5cm}
\begin{center}
\includegraphics[scale=0.2]{../jpg/RCvoltPolarPlotVectors.jpg}
\end{center}
\caption{\label{f112} Vectors are drawn in the polar plot of complex voltages in RC circuit. Note how they add up to 1V.}
\end{figure}





%\begin{figure}[htbp]
%\vspace*{-0.5cm}
%\begin{center}
%\includegraphics[scale=0.7]{../jpg/addingvectors.jpg}
%\end{center}
%\caption{\label{f113} Skeching phasors using vector magnitudes on paper.}
%\end{figure}
\end{explanation}



\end{example}





\end{document} 
