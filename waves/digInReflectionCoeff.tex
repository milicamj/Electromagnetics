\documentclass{ximera}  


\input{../preamble.tex}



 
\title{Reflection Coefficient} 
\author{Milica Markovic} 
\outcome{}
\begin{document}  
\begin{abstract}  

\end{abstract}  
\maketitle    



In this section, we will derive the equation for the reflection coefficient. The reflection coefficient relates the forward-going voltage with reflected voltage.

\section{Reflection coefficient at the load}

Equations \ref{eq:RCvtl}-\ref{eq:RCctl} represent the voltage and current on a lossless transmission line shown in Figure \ref{fig:RCTLCircuit}.

\begin{eqnarray}
\tilde{V}(z)=\tilde{V}_0^+ e^{-j \beta z} +\tilde{V}_0^- e^{j \beta z} \label{eq:RCvtl} \\ 
I(z)=\frac{\tilde{V}_0^+}{Z_0} e^{- j \beta z} - \frac{\tilde{V}_0^-}{Z_0} e^{j \beta z}\label{eq:RCctl}
\end{eqnarray}



\begin{figure}[htbp]
\begin{center}
\includegraphics[scale=0.3]{../jpg/trline.jpg}
%\strut\psfig{figure=trline.ps,width=3cm} \\
\end{center}
\caption{Transmission Line connects generator and the load.}
\label{fig:RCTLCircuit}
\end{figure}




We set up the z-axis so that the $z=0$ is at the load, and the generator is at $z=-l$. At $z=0$, the load impedance is connected. The definition of impedance is $Z=V/I$, therefore at the z=0 end of the transmission line, the voltage and current on the transmission line at that point have to obey boundary condition that the load impedance imposes.

\begin{eqnarray}
Z_L=\frac{V(0)}{I(0)} \nonumber 
\end{eqnarray}

Substituting z=0, the boundary condition, in Equations \ref{eq:RCvtl}-\ref{eq:RCctl}, we get Equations \ref{eq:RCvtl1}-\ref{eq:RCctl1}.

\begin{eqnarray}
\tilde{V}(0)=\tilde{V}_0^+ e^{-j \beta 0} +\tilde{V}_0^- e^{j \beta 0} = \tilde{V}_0^+ + \tilde{V}_0^- \label{eq:RCvtl1} \\ 
I(0)=\frac{\tilde{V}_0^+}{Z_0} e^{- j \beta 0} - \frac{\tilde{V}_0^-}{Z_0} e^{ j \beta 0} =\frac{\tilde{V}_0^+}{Z_0} - \frac{\tilde{V}_0^-}{Z_0}\label{eq:RCctl1}
\end{eqnarray}

Dividing the two above equations, we get the impedance at the load.


\begin{eqnarray}
Z_L=Z_0 \frac{\tilde{V}_0^+ + \tilde{V}_0^-}{\tilde{V}_0^+ - \tilde{V}_0^-}
\end{eqnarray}


We can now solve the above equation for $\tilde{V}_0^-$

\begin{eqnarray}
\frac{Z_L}{Z_0} (\tilde{V}_0^+ - \tilde{V}_0^-) = \tilde{V}_0^+ + \tilde{V}_0^- \nonumber \\
(\frac{Z_L}{Z_0}-1)\tilde{V}_0^+ =(\frac{Z_L}{Z_0}+1) \tilde{V}_0^- \nonumber \\
\frac{\tilde{V}_0^-}{\tilde{V}_0^+} = \frac{\frac{Z_L}{Z_0}-1  }{ \frac{Z_L}{Z_0}+1 }
\nonumber \\
\frac{\tilde{V}_0^-}{\tilde{V}_0^+} = \frac{Z_L -Z_0}{Z_L +Z_0}
\end{eqnarray}


\begin{definition}
$\Gamma_L=\frac{\tilde{V}_0^-}{\tilde{V}_0^+}= \frac{Z_L -Z_0}{Z_L +Z_0}$ is the voltage reflection
coefficient at the load. $\Gamma_L$ relates the reflected and incident voltage
phasor and the load $Z_L$ and transmission line impedance $Z_0$. The voltage reflection coefficient at the load is, in general, a complex number,
it has a magnitude and a phase $\Gamma_L=|\Gamma_L| e^{j \angle \Gamma_L}$.

\end{definition}

\subsection{Example}


 \begin{enumerate}
\item 100\,$\Omega$ transmission line is terminated in a series
connection of a 50\,$\Omega$ resistor and 10\,pF capacitor. The frequency
of operation is 100\,MHz. Find the voltage reflection coefficient.
\item For purely reactive load $Z_L=j 50 \Omega$, find the reflection
coefficient.
\end{enumerate}

\section{Voltage and Current on a transmission line}

Now that we related forward and reflected voltage on a transmission line with the reflection coefficient at the load, we can re-write the equations for the current and voltage on a transmission line as:

\begin{eqnarray}
\tilde{V}(z)= \tilde{V}_0^+ (e^{-j \beta z} + \Gamma_L  e^{j \beta z }  ) \label{eq:vtlfin} \\
I(z)=   \frac{\tilde{V}_0^+}{Z_0}  (e^{-j \beta z} - \Gamma  e^{j \beta z}  ) \label{eq:itlfin}
\end{eqnarray}

We see that if we know the length of the line, line type, the load impedance, and the transmission line impedance, we can calculate all variables above, except for  $\tilde{V}_0^+ $. In the following chapters, we will derive the equation for the forward going voltage at the load, but first, we will look at little more at the various reflection coefficients on a transmission line. 

\section{Reflection coefficient anywhere on the line}

Equations \ref{eq:RCvtl}-\ref{eq:RCctl}  can be concisely written as 

\begin{eqnarray}
\tilde{V}(z) =\tilde{V}(z)^+ + \tilde{V}(z)^- \\
\tilde{I}(z) =\tilde{I}(z)^+ + \tilde{I}(z)^- 
\end{eqnarray}

Where $\tilde{V}(z)^+$ is the forward voltage anywhere on the line, $\tilde{V}(z)^-$ is reflected voltage anywhere on the line, $\tilde{I}(z)^+$ is the forward current anywhere on the line, and  $\tilde{I}(z)^-$ is the reflected current anywhere on the line.

We can then define a reflection coefficient anywhere on the line as

\begin{definition}
$\Gamma(z)=\frac{\tilde{V}(z)^-}{\tilde{V}(z)^+} = \frac{\tilde{V}_0^- e^{j \beta z}}{\tilde{V}_0^+ e^{-j \beta z}} = \frac{\tilde{V}_0^-}{\tilde{V}_0^+} e^{2j \beta z} $ is a voltage reflection coefficient anywhere on the line. $\Gamma(z)$ relates the reflected and incident voltage
phasor at any z. 

Since we already defined $\Gamma_L=\frac{\tilde{V}_0^-}{\tilde{V}_0^+}$ as the reflection coefficient at the load, we can now simplify the general reflection coefficient as 

\begin{equation}
\Gamma(z)=\Gamma_L e^{2j \beta z}
\end{equation}

It is important to remember that we defined points between the generator and the load as the negative z-axis. If the line length is, for example, l\,m long, the generator is then at z=-l\,m, and the load at z=0. To find the reflection coefficient at some distance $l/2$\,m away from the load,  at $z=-l/2$\,m, the equation for the reflection coefficient will be


\begin{equation}
\Gamma(z=-l/2)=\Gamma_L e^{-2j \beta l/2}
\end{equation}

\end{definition}

Since we already defined the reflection coefficient at the load, the reflection at any point on the line $z=-l$ is


\begin{eqnarray}
\Gamma(z=-l)=\Gamma_L e^{-2j \beta l} \\
\Gamma(z=-l)=|\Gamma_L| e^{j ( \angle \Gamma_L - 2 \beta l) }
\end{eqnarray}



\section{Reflection coefficient at the input of the transmission line}

Using the reasoning above, the reflection coefficient at the input of the line whose length is $l$ is
\begin{equation}
\Gamma(z=-l)=\Gamma_{in}=\Gamma_L e^{-2j \beta l}
\end{equation}

\begin{example}
The reflection coefficient at the load is $\Gamma_L=0.5 e^{j60^0}$. Find the input reflection coefficient if the electrical length of the line is $\beta l= 45^0$.

\begin{explanation}
The reflection coefficient at the input of the line is $\Gamma(z=-l)=\Gamma_{in}=|\Gamma_L| e^{j ( \angle \Gamma_L - 2 \beta l) }$. 

We substitute the expression for $\Gamma_L=0.5 e^{j60^0}$ and $\beta l= 90^0$, we get the reflection coefficient at the input of the line $\Gamma_{in}=0.5 e^{-j30 }$.

\end{explanation}

\end{example}



\end{document} 
