\documentclass{ximera}  


\input{../preamble.tex}



 
\title{Waves on Transmission Lines} 
\author{Milica Markovic} 
\outcome{Apply phasor transformation to a time-domain equation to obtain frequency-domain equation.}
\begin{document}  
\begin{abstract}  

\end{abstract}  
\maketitle    











\section{Relating forward and backward current and voltage waves on
the transmission line}


In equations \ref{eq4}-\ref{eq5} $\tilde{V}_0^+$ and $\tilde{V}_0^-$ are the phasors of forward and
reflected going voltage waves, and $\tilde{I}_0^+$ and $\tilde{I}_0^-$ are the phasors of forward and
reflected current waves. This section will relate the phasors of voltage and
current waves with transmission-line impedance.


\begin{eqnarray}
\tilde{V}(z)=\tilde{V}_0^+ e^{-\gamma z} + \tilde{V}_0^- e^{\gamma z}\label{eq4} \\
I(z)=\tilde{I}_0^+ e^{-\gamma z} + \tilde{I}_0^- e^{\gamma z}\label{eq5}
\end{eqnarray}




When substitute the voltage wave equation into Telegrapher's  Equations
\ref{te11}. The equation is repeated here  Eq.\ref{te12new}-\ref{te12new1}.




\begin{eqnarray}
-\frac{\partial \tilde{V}(z)}{\partial z} = (R+j\omega L) I(z) \label{te12new} \\
\gamma \tilde{V}_0^+ e^{-\gamma z} - \gamma \tilde{V}_0^- e^{\gamma z} = (R+ j \omega
L) I(z) \label{te12new1}
\end{eqnarray}

We now rearrange Eq.\ref{te12new1}

\begin{eqnarray}
I(z)=\frac{\gamma}{R+j\omega L} ( \tilde{V}_0^+ e^{-\gamma z} + \tilde{V}_0^-
 e^{\gamma z})  \nonumber  \\
I(z)=\frac{\gamma \tilde{V}_0^+}{R+ j \omega L} e^{-\gamma z} - \frac{\gamma \tilde{V}_0^-}{R+ j \omega L} e^{\gamma z} \label{eq3}
\end{eqnarray}

Now we compare Eq.\ref{eq3} with the Eq.\ref{eq5}. For two transcendental equations to be equal, the coefficients next to exponential terms have to be the same.

\begin{eqnarray}
\tilde{I}_0^+=\frac{\gamma \tilde{V}_0^+}{R+ j \omega L} \nonumber  \\
\tilde{I}_0^-= - \frac{\gamma \tilde{V}_0^-}{R+ j \omega L} \nonumber
\end{eqnarray}

We can define the characteristic impedance of a transmission line as
the ratio of the voltage to the current amplitude of the forward-going
wave.


\begin{eqnarray}
Z_0=\frac{\tilde{V}_0^+}{ \tilde{I}_0^+} \nonumber   \\ \nonumber
Z_0=\frac{R+j\omega L}{\gamma} \nonumber   \\ \nonumber
Z_0=\sqrt{\frac{R+j\omega L}{G+ j\omega C}}
\end{eqnarray}






\section{Voltage Reflection Coefficient, Lossless Case}

The equations for the voltage and current on the transmission line we
derived so far are

\begin{eqnarray}
\tilde{V}(z)=\tilde{V}_0^+ e^{-j \beta z} +\tilde{V}_0^- e^{j \beta z} \label{vtl} \\ \label{ctl}
I(z)=\frac{\tilde{V}_0^+}{Z_0} e^{- j \beta z} - \frac{\tilde{V}_0^-}{Z_0} e^{j \beta z}
\end{eqnarray}



\begin{figure}[htbp]
\begin{center}
\includegraphics[scale=0.3]{../jpg/trline.jpg}
%\strut\psfig{figure=trline.ps,width=3cm} \\
\end{center}
\caption{Transmission Line connects generator and the load.}
\label{wind1}
\end{figure}





At $z=0$ the impedance of the load has to be

\begin{eqnarray}
Z_L=\frac{V(0)}{I{0}} \nonumber 
\end{eqnarray}

Substitute the boundary condition in Eq.\ref{vtl}

\begin{eqnarray}
Z_L=Z_0 \frac{\tilde{V}_0^+ + \tilde{V}_0^-}{\tilde{V}_0^+ - \tilde{V}_0^-}
\end{eqnarray}


We can now solve the above equation for $\tilde{V}_0^-$

\begin{eqnarray}
\frac{Z_L}{Z_0} (\tilde{V}_0^+ - \tilde{V}_0^-) = \tilde{V}_0^+ + \tilde{V}_0^- \nonumber \\
(\frac{Z_L}{Z_0}-1)\tilde{V}_0^+ =(\frac{Z_L}{Z_0}+1) \tilde{V}_0^- \nonumber \\
\frac{\tilde{V}_0^-}{\tilde{V}_0^+} = \frac{\frac{Z_L}{Z_0}-1  }{ \frac{Z_L}{Z_0}+1 }
\nonumber \\
\frac{\tilde{V}_0^-}{\tilde{V}_0^+} = \frac{Z_L -Z_0}{Z_L +Z_0}
\end{eqnarray}

The quantity $\frac{\tilde{V}_0^-}{\tilde{V}_0^+}$ is called voltage reflection
coefficient $\Gamma$. $\Gamma$ relates the reflected and incident voltage
phasor. The voltage reflection coefficient is, in general, a complex number,
it has a magnitude and a phase.



\subsubsection{Examples}


 \begin{enumerate}
\item 100\,$\Omega$ transmission line is terminated in a series
connection of a 50\,$\Omega$ resistor and 10\,pF capacitor. The frequency
of operation is 100\,MHz. Find the voltage reflection coefficient.
\item For purely reactive load $Z_L=j X_L$ find the reflection
coefficient.
\end{enumerate}

The end of this lecture is spent in the lab making a Matlab program to
make a movie of a wave moving left and right.



\end{document} 
