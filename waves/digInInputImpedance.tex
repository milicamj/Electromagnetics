\documentclass{ximera}  


\input{../preamble.tex}



 
\title{Input impedance of a transmission line} 
\author{Milica Markovic} 
\outcome{Derive and calculate the input impedance of a transmission line.}


\begin{document}  
\begin{abstract}  

\end{abstract}  
\maketitle    



Again, we will look at a transmission line circuit in Figure \ref{fig:IITRLine} to find the input impedance on a transmission line.

\begin{figure}[htbp]
\begin{center}
\includegraphics[scale=0.3]{../jpg/trline.jpg}
%\strut\psfig{figure=trline.ps,width=3cm} \\
\end{center}
\caption{Transmission Line connects generator and the load.}
\label{fig:IITRLine}
\end{figure}




The equations for the voltage and current anywhere (any z) on a transmission line  are


\begin{eqnarray}
\tilde{V}(z)= \tilde{V}_0^+ (e^{-j \beta z} + \Gamma_L  e^{j \beta z }  ) \label{eq:vtlfin} \\
I(z)=   \frac{\tilde{V}_0^+}{Z_0}  (e^{-j \beta z} - \Gamma  e^{j \beta z}  ) \label{eq:itlfin}
\end{eqnarray}


The voltage and current equations at the generator $z=-l$ are:

\begin{eqnarray}
\tilde{V}_{in}=\tilde{V}(z=-l)= \tilde{V}_0^+ (e^{j \beta l} + \Gamma_L  e^{-j \beta l }  )  \\
\tilde{I}_{in}=I(z=-l)=   \frac{\tilde{V}_0^+}{Z_0}  (e^{j \beta l} - \Gamma  e^{-j \beta l}  ) 
\end{eqnarray}


\section{Input impedance as a function of reflection coefficient}

The input impedance is defined as $Z_{in}=\frac{V_{in}}{I_{in}}$. Since the line length is $l$, the input impedance is





\begin{eqnarray}
Z_{in}=\frac{\tilde{V}_0^+ (e^{-j \beta z} + \Gamma_L  e^{j \beta z }  )}{\frac{\tilde{V}_0^+}{Z_0}  (e^{-j \beta z} - \Gamma_L  e^{j \beta z}  )}
\end{eqnarray}

If we cancel common terms, we get

\begin{eqnarray}
Z_{in}=Z_0 \frac{(e^{-j \beta z} + \Gamma_L  e^{j \beta z }  )}{  (e^{-j \beta z} - \Gamma_L  e^{j \beta z}  )} \label{eq:theSecondway}
\end{eqnarray}

Now we can take $e^{-j \beta z}$ in front of parenthesis from both numerator and denominator and then cancel it.


\begin{eqnarray}
Z_{in}=Z_0 \frac{1 + \Gamma_L  e^{ 2j \beta z }  }{  1 - \Gamma_L  e^{2j \beta z}  }
\end{eqnarray}

We have previously defined the reflection coefficient at the transmission line's input as $\Gamma_{in}=\Gamma_L  e^{ 2 j \beta z }$. The final equation for the input impedance is therefore




\begin{eqnarray}
Z_{in}=Z_0 \frac{1 + \Gamma_{in}  }{  1 - \Gamma_{in}  }
\end{eqnarray}

\subsection{Input impedance as a function of load impedance}

If we now look back at the Equation \ref{eq:theSecondway}, here we can also  use Euler's formula $e^{j \beta z} = \cos  (\beta z) + j\sin  (\beta z)  $, and the equation for the reflection coefficient at the load $\Gamma_L = \frac{Z_L-Z_0}{Z_L+Z_0}$ we find the input impedance of the line as shown below.


\begin{eqnarray}
Z_{in}= Z_0 \frac{Z_L+ j Z_0 \tan \beta l}{Z_0+ j Z_L \tan \beta l} 
\end{eqnarray}

This equation will be soon become obsolete when we learn how to use the Smith Chart.

\begin{example}

Find the input impedance if the load impedance is $Z_L=0 \Omega$, and the electrical length of the line is $\beta l = 90^0$.

\begin{explanation}
Since the load impedance is a short circuit, and the angle is $90^0$ the equation simplifies to $Z_{in}=  j Z_0 \tan \beta l = \infty$.
\end{explanation}

\end{example}



When we find the input impedance, we can replace the transmission line and the load, as shown in Figure \ref{fig:IITRLineEqCirc}. In the next section, we will use input impedance to find the forward going voltage on a transmission line.

\begin{figure}[htbp]
\begin{center}
\includegraphics[scale=1]{../jpg/trlineEqCirc.jpg}
%\strut\psfig{figure=trline.ps,width=3cm} \\
\end{center}
\caption{Transmission Line connects generator and the load.}
\label{fig:IITRLineEqCirc}
\end{figure}



\end{document} 
