\documentclass{ximera}  


\input{../preamble.tex}



 
\title{Wave Equation} 
\author{Milica Markovic} 
\outcome{Apply phasor transformation to a time-domain equation to obtain frequency-domain equation.}
\begin{document}  
\begin{abstract}  

\end{abstract}  
\maketitle    




\section{ Wave equation on a transmission line}\label{telegrapher}

In this section, we will derive the expression for voltage and current
along a transmission line. This expression will have two variables, time $t$, and space $z$. 
So far, we have only seen voltages and currents as a function of time, because all circuit elements seen so far are lumped elements. In distributed systems,
we want to derive the equations for voltage and current for the case when the transmission
line is longer than the fraction of a wavelength.  To make sure that we
do not encounter any transmission line effects to start with, we can
look at the piece of a transmission line that is much smaller than
the fraction of a wavelength. In other words, we cut the transmission
line into small pieces to make sure there are no transmission line
effects, as the pieces are shorter than the fraction of a wavelength. We then represent each piece with
an equivalent circuit, as shown in Figure \ref{lineeqc} (a). 
To derive expressions for current and voltage on the transmission line, we will use the following five-step plan

\begin{enumerate}
\item Look at an infinitesimal length of a transmission line $\Delta z$.  

\item Represent that piece with an equivalent circuit. 

\item Write KCL, KVL for the piece in the time domain (we get
differential equations)

\item Apply phasors (equations become linear)

\item Solve the linear system of equations to get the expression for
the voltage and current on the transmission line as a function of $z$.

\end{enumerate}


Look at a small piece of a transmission line and represented it with an equivalent circuit. What is
modeled by the circuit elements?



\begin{figure}[htbp]
\begin{center}
\includegraphics[scale=0.3]{../jpg/Coaxtl.jpg}
\caption{Coaxial cable is cut in short pieces.}
\label{lineeqcPieces}
\end{center}
\end{figure}

\begin{figure}[htbp]
\begin{center}
\includegraphics[scale=0.3]{../jpg/Equivalent_Circuit_of_Transmission_Line.jpg}
\caption{Equivalent circuit of a section of transmission line.}
\label{lineeqcOnePiece}
\end{center}
\end{figure}

\begin{figure}[htbp]
\begin{center}
\includegraphics[scale=0.4]{../jpg/tlmadeupofcircuits.jpg}
\end{center}
\caption{Equivalent circuit of transmission line.}
\label{lineeqc}
\end{figure}



Write  KVL and KCL equations for the circuit above.

KVL
\begin{eqnarray}
-v(z,t) + R \, \Delta z \, i(z,t) + L \,\Delta z \,\frac{\partial
 i(z,t)}{\partial t} + v(z+ \Delta z,t) = 0 \nonumber
\end{eqnarray}

KCL

\begin{eqnarray}
i(z,t)=i(z+\Delta z)+ i_{CG}(z+\Delta z,t) \nonumber   \\
i(z,t)=i(z+\Delta z)+ G \, \Delta z\, v(z+\Delta z,t)+C\, \Delta z
\frac{\partial v(z+\Delta z,t)}{\partial t} \nonumber
\end{eqnarray}


Rearrange the KCL and KVL Equations \ref{te1kvl1}, \ref{te1kc11} and divide them with
$\Delta z$.  Equations \ref{te2kvl1}, \ref{te1kc21}.
let $\Delta z \to 0$ and recognize the definition of the
derivative Equations, \ref{te2kvl111}, \ref{te1kc211}.

KVL
\begin{eqnarray}
-( v(z+ \Delta z ,t)- v(z,t))=R \, \Delta z  \, i(z,t)+L  \,  \Delta z
 \frac{\partial i(z,t)}{\partial t} \label{te1kvl1}  \\ 
 -\frac{ v(z+ \Delta z ,t)- v(z,t)}{\Delta z}=R  \, i(z,t)+L  \, 
 \frac{\partial i(z,t)}{\partial t}  \label{te2kvl1} \\
\lim_{\Delta z \to 0} \{ -\frac{ v(z+ \Delta z ,t)- v(z,t)}{\Delta z}\}=  \lim_{\Delta z \to 0} \{   R  \,  i(z,t)+L  \, 
 \frac{\partial i(z,t)}{\partial t} \} \label{te2kvl111} \\
-\frac{v(z,t) }{\partial z}=R i(z,t)+L 
 \frac{\partial i(z,t)}{\partial t} \label{te3kvl1}
\end{eqnarray}

KCL

\begin{eqnarray}
-( i(z+ \Delta z ,t)- i(z,t))= G  \, \Delta z  \, v(z+\Delta z,t)+ C \, \Delta z
\frac{\partial v(z+\Delta z,t)}{\partial t} \label{te1kc11} \\
-\frac{ i(z+ \Delta z ,t)- i(z,t)}{\Delta z}= G  \, v(z+\Delta z,t)+C \, 
\frac{\partial v(z+\Delta z,t)}{\partial t} \label{te1kc21} \\
\lim_{\Delta z \to 0} \{-\frac{ i(z+ \Delta z ,t)- i(z,t)}{\Delta z} \}= \lim_{\Delta z \to 0} \{ G  \, v(z+\Delta z,t)+C \, 
\frac{\partial v(z+\Delta z,t)}{\partial t} \} \label{te1kc211} \\
-\frac{i(z,t) }{\partial z}= G  \, v(z+\Delta z,t)+C \, 
\frac{\partial v(z+\Delta z,t)}{\partial t} \label{te1kc31}
\end{eqnarray}



We just derived Telegrapher's equations in time-domain:



\begin{eqnarray}
-\frac{v(z,t) }{\partial z}=R  \, i(z,t)+L  \, 
 \frac{\partial i(z,t)}{\partial t} \nonumber  \\ \nonumber
-\frac{i(z,t) }{\partial z}= G \,  v(z+\Delta z,t)+C \, 
\frac{\partial v(z+\Delta z,t)}{\partial t} 
\end{eqnarray}


Telegrapher's equations are two differential equations with two unknowns, $i(z, t)$, $v(z, t)$. It is not
impossible to solve them; however, we would prefer to have linear algebraic equations. We then express time-domain variables as phasors.

\begin{eqnarray}
v(z,t)=Re\{ \tilde{V}(z) e^{j \omega t} \} \nonumber \\
i(z,t)=Re\{ \tilde{I}(z) e^{j \omega t} \} \nonumber
\end{eqnarray}

 $\tilde{V}(z),\tilde{I}(z) $ are the voltage, and current anywhere on the line, and they depend on the position on the line $z$. The Telegrapher's equations in phasor form are


\begin{eqnarray}
-\frac{\partial \tilde{V}(z)}{\partial z} = (R+j\omega L) \tilde{I}(z) \label{te11}\\
-\frac{\partial \tilde{I}(z)}{\partial z} = (G+j\omega C) \tilde{V}(z) \label{te121}
\end{eqnarray}



Two equations, two unknowns. To solve these equations, we first
take a derivative of both equations with respect to z. 

\begin{eqnarray}
-\frac{\partial^2 \tilde{V}(z)}{\partial z^2}=  (R+j\omega L) \frac{\partial
 \tilde{I}(z)}{\partial z}  \label{teleg3} \\
-\frac{\partial^2 \tilde{I}(z)}{\partial z^2}=  (G+j\omega C) \frac{\partial
 \tilde{V}(z)}{\partial z} \label{teleg4}
\end{eqnarray}

Rearange the previous equations:



\begin{eqnarray}
- \frac{1}{ (R+j\omega L)} \frac{\partial \tilde{I}(z)}{\partial z}= \frac{\partial^2
  \tilde{V}(z)}{\partial z^2} \label{te51} \\
-\frac{1}{ (G+j\omega C)} \frac{\partial \tilde{V}(z)}{\partial z}= \frac{\partial^2
  \tilde{I}(z)}{\partial z^2} \label{te61}
\end{eqnarray}

Substitute  Eq.\ref{te51} into  Eq.\ref{te121}
and Eq.\ref{te61} into Eq.\ref{te11} and we get

\begin{eqnarray}
-\frac{\partial^2 \tilde{V}(z)}{\partial z^2}=(G+j\omega C)(R+j\omega L) \tilde{V}(z) \label{teleg1} \\
-\frac{\partial^2 \tilde{I}(z)}{\partial z^2}= (G+j\omega C)  (R+j\omega L) \label{teleg2}
\tilde{I}(z) 
\end{eqnarray}

Or if we rearrange


\begin{eqnarray}
\frac{\partial^2 \tilde{V}(z)}{\partial z^2} -(G+j\omega C)(R+j\omega L)
 \tilde{V}(z)=0  \label{we11} \\ 
\frac{\partial^2 I(z)}{\partial z^2}- (G+j\omega C)  (R+j\omega L)
I(z)=0 \label{we21}
\end{eqnarray}

The above Eq.\ref{we11} and Eq.\ref{we21} are called the wave equation, and they represent
current and voltage wave on a transmission line. $\gamma=(G+j\omega
C)(R+j\omega L)$ is the complex propagation constant. This constant
has a real and an imaginary part.

\begin{eqnarray}
\gamma= \alpha + j \beta \nonumber
\end{eqnarray}

where $\alpha$ is the attenuation constant and $\beta$ is the phase
constant.

\begin{eqnarray}
\alpha=Re\{ \sqrt{(G+j\omega C)  (R+j\omega L)  }  \} \nonumber \\ \nonumber
\beta = Im\{ \sqrt{(G+j\omega C)  (R+j\omega L)  }  \}
\end{eqnarray}

The general solution of the second order differential equations with constant coefficients
Equations \ref{we11} - \ref{we21}   is:

\begin{eqnarray}
\tilde{V}(z)=\tilde{V}_0^+ e^{-\gamma z} + \tilde{V}_0^- e^{\gamma z} \nonumber \\ \nonumber
\tilde{I}(z)=\tilde{I}_0^+ e^{-\gamma z} + \tilde{I}_0^- e^{\gamma z}
\end{eqnarray}

In this equation $\tilde{V_0}^+$ and $\tilde{V_0}^-$ are the {\bf
phasors} of forward and
reflected voltage waves at the load (where z=0), and $\tilde{I_0}^+$ and $\tilde{I_0}^-$ are the phasors of forward and
reflected current wave at the load (where z=0).These voltages and currents are also phasors and have a constant magnitude and phase in a specific circuit, for example $\tilde{V_0}^+=|\tilde{V_0}^+| e^\Phi=4e^{25^0}$, and $\tilde{I_0}^+=|\tilde{I_0}^+| e^\Phi=5e^{-40^0}$.
We can get the time-domain expression for the current and voltage on the
transmission line by multiplying the phasor of the voltage and current with $e^{j \omega t}$ and taking the real part of it.

\begin{eqnarray}
v(t)=Re\{ (\tilde{V}_0^+  e^{(-\alpha - j \beta) z} + \tilde{V}_0^- e^{(\alpha + j
 \beta) z})e^{j \omega t} \}  \nonumber \\ 
v(t)=|\tilde{V}_0^+| e^{ - \alpha z} \cos(\omega t - \beta z + \angle \tilde{V}_0^+)+
|\tilde{V}_0^-|e^{\alpha z} \cos(\omega t + \beta z + \angle \tilde{V}_0^-) \label{tdeq}
\end{eqnarray}

We will look at the Matlab program in class to see that if the signs of the $\omega t$ and
$\beta z$ are the same the wave moves in the forward $+z$
direction. If the signs of $\omega t$ and $\beta z$ are opposite, the
wave moves in the $-z$ direction.

In the next several sections, we will look at how to find the constants $\beta$, $\tilde{V}_0^+  $, $ \tilde{V}_0^-$, $\tilde{I}_0^+ $, $\tilde{I}_0^-$. In order to find the constants, we will introduce the concepts of transmission line impedance $Z_0$, reflection coefficient $\Gamma(z)$, input impedance $Z_{in}$. 


\section{Transmission-line parameters R, G, C, and L}

To find the propagation constant, we need to know how to find the transmission-line parameters R, G, C, and L. Equations for R, G, C, and L for a coaxial cable are given in the table below. 

\begin{center}
\begin{tabular}{|c|c|c|c|c|} \hline
Transmission-line    & R  & G & C &L     \\  \hline       
Coaxial Cable       & $\frac{R_{sd}}{2 \pi} \left(\frac{1}{a} +  \frac{1}{b} \right)$   & $\frac{2 \pi \sigma}{\ln b/a}$ &$\frac{2 \pi \epsilon}{\ln b/a}$& $\frac{\mu}{2 \pi } \ln b/a$                             \\ \hline    
\end{tabular}
\end{center}

Where $R_{sd} = \sqrt{\pi f \mu_m/\sigma_m}$ is the resistance associated with skin-depth. $f$ is the frequency of the signal, $\mu_m$ is the magentic permeability of conductors, $\sigma_c$ is the conductivity of conductors.

\end{document} 
