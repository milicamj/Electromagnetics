\documentclass{ximera}  


\input{../preamble.tex}



 
\title{Forward voltage on a transmission line} 
\author{Milica Markovic} 
\outcome{}
\begin{document}  
\begin{abstract}  

\end{abstract}  
\maketitle    






\begin{figure}[htbp]
\begin{center}
\includegraphics[scale=0.3]{../jpg/trline.jpg}
%\strut\psfig{figure=trline.ps,width=3cm} \\
\end{center}
\caption{Transmission Line connects generator and the load.}
\label{fig:TRLine}
\end{figure}


\begin{explanation}

The equations for the voltage and current anywhere (any z) on a transmission line  are


\begin{eqnarray}
\tilde{V}(z)= \tilde{V}_0^+ (e^{-j \beta z} - \Gamma_L  e^{j \beta z }  ) \label{eq:vtlfin} \\
I(z)=   \frac{\tilde{V}_0^+}{Z_0}  (e^{-j \beta z} - \Gamma_L  e^{j \beta z}  ) \label{eq:itlfin}
\end{eqnarray}

\section{Forward voltage phasor as a function of load impedance}


\begin{eqnarray}
\tilde{V}_0^+= \frac{\tilde{V}_g Z_{in}}{Z_g + Z_{in}} \frac{1}{e^{j \beta l} + \Gamma e^{-j \beta l}} \\
\end{eqnarray}




\section{Forward voltage phasor as a function of input reflection coefficient}


\begin{equation}
\tilde{V}_0^+=V_g e^{-j \beta l} \frac{Z_0}{Z_0 (1+\Gamma_{in}) +Z_g (1-\Gamma_{in})}
\end{equation}

\section{Special case - forward voltage when the generator and transmission-line impedance are equal}

Because the generator's impedance is equal to the transmission line impedance, we will use the second equation. When $Z_g=Z_0$ we see that the denominator simplifies into $Z_0 (1+\Gamma{in}) +Z_g (1-\Gamma_{in}) = Z_0 (1+\Gamma{in}+1-1+\Gamma{g\in})$ and we can further simplify the fraction to get the final value of $\tilde{V}_0^+=\frac{V_g}{2} e^{-j \beta l}$. 





\end{explanation}








\end{document} 
