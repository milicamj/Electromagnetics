\documentclass{ximera}  


\input{../preamble.tex}



 
\title{Forward voltage on a transmission line} 
\author{Milica Markovic} 
\outcome{Calculate  phasors of forward going voltage at the load.}
\begin{document}  
\begin{abstract}  

\end{abstract}  
\maketitle    



Again, we will look at a transmission line circuit in Figure \ref{fig:FVTRLine} to find the input impedance on a transmission line.





\begin{figure}[htbp]
\begin{center}
\includegraphics[scale=0.3]{../jpg/trline.jpg}
%\strut\psfig{figure=trline.ps,width=3cm} \\
\end{center}
\caption{Transmission Line connects generator and the load.}
\label{fig:FVTRLine}
\end{figure}





The equations for the voltage and current anywhere (any z) on a transmission line  are


\begin{eqnarray}
\tilde{V}(z)= \tilde{V}_0^+ (e^{-j \beta z} - \Gamma_L  e^{j \beta z }  ) \label{eq:FVvtlfin} \\
I(z)=   \frac{\tilde{V}_0^+}{Z_0}  (e^{-j \beta z} - \Gamma_L  e^{j \beta z}  ) \label{eq:FVitlfin}
\end{eqnarray}


Using the equations from the previous section, we can replace the transmission line with its input impedance, Figure \ref{fig:FVTRLineEqCirc}.



\begin{figure}[htbp]
\begin{center}
\includegraphics[scale=0.3]{../jpg/trlineEqCirc.jpg}
%\strut\psfig{figure=trline.ps,width=3cm} \\
\end{center}
\caption{Transmission Line connects generator and the load.}
\label{fig:FVTRLineEqCirc}
\end{figure}



\section{Forward voltage phasor as a function of load impedance}


From Figure \ref{fig:FVTRLineEqCirc}, we can find the input voltage on a transmission line using the voltage divider.

\begin{equation}
\tilde{V}_{in}= \frac{Z_{in}}{Z_{in}+Z_g} \tilde{V}_g
\end{equation} 

Using Equation \ref{eq:FVitlfin}, we can also find the input voltage.
The input voltage equation at the generator $z=-l$ is:

\begin{eqnarray}
\tilde{V}_{in}=\tilde{V}(z=-l)= \tilde{V}_0^+ (e^{j \beta l} + \Gamma_L  e^{-j \beta l } )
\end{eqnarray}

Since these two equations represent the same input voltage we can make them equal.
\begin{equation}
\tilde{V}_0^+ (e^{j \beta l} + \Gamma_L  e^{-j \beta l }) = \frac{Z_{in}}{Z_{in}+Z_g} \tilde{V}_g
\end{equation}

Rearranging the equation, we find $\tilde{V}_0^+$.

\begin{eqnarray}
\tilde{V}_0^+= \frac{\tilde{V}_g Z_{in}}{Z_g + Z_{in}} \frac{1}{e^{j \beta l} + \Gamma_L e^{-j \beta l}} \\
\end{eqnarray}




\section{Forward voltage phasor as a function of input reflection coefficient}


There is another way to find the input impedance as a function of the input reflection coefficient.


We write KVL for the circuit in Figure \ref{fig:FVTRLineEqCirc}.

\begin{equation}
\tilde{V}_g=Z_g \tilde{I}_{in} + \tilde{V}_{in} \label{eq:FVKVL}
\end{equation}


Using Equations \ref{eq:FVitlfin}-\ref{eq:FVitlfin}, we can also find the input voltage and current.
Input voltage and current equation at the generator $z=-l$ are:


\begin{eqnarray}
\tilde{V}_{in}=\tilde{V}(z=-l)= \tilde{V}_0^+ (e^{j \beta l} + \Gamma_L  e^{-j \beta l } ) \\
\tilde{I}_{in}=I(z=-l)=   \frac{\tilde{V}_0^+}{Z_0}  (e^{j \beta l} - \Gamma  e^{-j \beta l}  ) 
\end{eqnarray}


Substituting these two equations in Equation \ref{eq:FVKVL} we get


\begin{equation}
\tilde{V}_g=Z_g \frac{\tilde{V}_0^+}{Z_0}  (e^{j \beta l} - \Gamma_L  e^{-j \beta l}  ) + \tilde{V}_0^+ (e^{j \beta l} + \Gamma_L  e^{-j \beta l } ) 
\end{equation}


We can re-write this equation as follows.



\begin{equation}
\tilde{V}_g=Z_g e^{j \beta l} \frac{\tilde{V}_0^+}{Z_0}  (1 - \Gamma_L  e^{-2j \beta l}  ) + \tilde{V}_0^+ e^{j \beta l} (1 + \Gamma_L  e^{-2j \beta l } ) 
\end{equation}

Using that 
 $\Gamma_{in} = \Gamma_L e^{-2 j \beta l} $ is the input reflection coefficient, and multiplying through with $Z_0$.


\begin{equation}
\tilde{V}_g Z_0=Z_g e^{j \beta l} \frac{\tilde{V}_0^+}{1}  (1 - \Gamma_{in} ) + \tilde{V}_0^+ Z_0 e^{j \beta l} (1 + \Gamma_{in}) 
\end{equation}


Rearranging the equation, we get $\tilde{V}_0^+$

\begin{equation}
\tilde{V}_0^+=\tilde{V}_g e^{-j \beta l} \frac{Z_0}{Z_0 (1+\Gamma_{in}) +Z_g (1-\Gamma_{in})}
\end{equation}

 $\Gamma_{in} = \Gamma_L e^{-2 j \beta l} $ is the input reflection coefficient.



\section{Special case - forward voltage when the generator and transmission-line impedance are equal}

Because the generator's impedance is equal to the transmission line impedance, we will use the second equation. When $Z_g=Z_0$ we see that the denominator simplifies into $Z_0 (1+\Gamma{in}) +Z_g (1-\Gamma_{in}) = Z_0 (1+\Gamma{in}+1-1+\Gamma{g\in})$ and we can further simplify the fraction to get the final value of $\tilde{V}_0^+=\frac{V_g}{2} e^{-j \beta l}$. 











\end{document} 
