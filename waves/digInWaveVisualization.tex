\documentclass{ximera}  


\input{../preamble.tex}



 
\title{Visualization of waves on lossless  transmission lines} 
\author{Milica Markovic} 
\outcome{Apply phasor transformation to a time-domain equation to obtain frequency-domain equation.}
\begin{document}  
\begin{abstract}  

\end{abstract}  
\maketitle    


\begin{eqnarray}
\tilde{V}(z)=\tilde{V}_0^+ e^{-\gamma z} + \tilde{V}_0^- e^{\gamma z} \nonumber \\ \nonumber
\tilde{I}(z)=\tilde{I}_0^+ e^{-\gamma z} + \tilde{I}_0^- e^{\gamma z}
\end{eqnarray}

In this equation $\tilde{V}(z)$ is the total voltage anywhere on the line (at any point z), $\tilde{I}(z)$ is the total current anywhere on the line (at any point z), $\tilde{V_0}^+$ and $\tilde{V_0}^-$ are the {\bf
phasors} of forward and
reflected voltage waves at the load (where z=0), and $\tilde{I_0}^+$ and $\tilde{I_0}^-$ are the phasors of forward and
reflected current wave at the load (where z=0).These voltages and currents are also phasors and have a constant magnitude and phase in a specific circuit, for example $\tilde{V_0}^+=|\tilde{V_0}^+| e^\Phi=4e^{25^0}$, and $\tilde{I_0}^+=|\tilde{I_0}^+| e^\Phi=5e^{-40^0}$.
We can get the time-domain expression for the current and voltage on the
transmission line by multiplying the phasor of the voltage and current with $e^{j \omega t}$ and taking the real part of it.

\begin{eqnarray}
v(t)=Re\{ (\tilde{V}_0^+  e^{(-\alpha - j \beta) z} + \tilde{V}_0^- e^{(\alpha + j
 \beta) z})e^{j \omega t} \}  \nonumber \\ 
v(t)=|\tilde{V}_0^+| e^{ - \alpha z} \cos(\omega t - \beta z + \angle \tilde{V}_0^+)+
|\tilde{V}_0^-|e^{\alpha z} \cos(\omega t + \beta z + \angle \tilde{V}_0^-) \label{tdeq}
\end{eqnarray}

We will look at the Matlab program in class to see that if the signs of the $\omega t$ and
$\beta z$ are the same the wave moves in the forward $+z$
direction. If the signs of $\omega t$ and $\beta z$ are opposite, the
wave moves in the $-z$ direction.

In the next several sections, we will look at how to find the constants $\beta$, $\tilde{V}_0^+  $, $ \tilde{V}_0^-$, $\tilde{I}_0^+ $, $\tilde{I}_0^-$. In order to find the constants, we will introduce the concepts of transmission line impedance $Z_0$, reflection coefficient $\Gamma(z)$, input impedance $Z_{in}$. 

\begin{example}

We will show next that if  the signs of the $\omega t$ and
$\beta z$ are the same, as in Equation \ref{bck1}, the wave moves in the forward $+z$
direction. If the signs of $\omega t$ and $\beta z$ are opposite, as in Equation \ref{fwd1}, the
wave moves in the $-z$ direction. In order to see this, we will visualize Equations \ref{fwd1} and \ref{bck1} using Matlab code below.

\begin{eqnarray}
v_f(t)=|\tilde{V}_0^+| e^{ - \alpha z} \cos(\omega t - \beta z + \angle \tilde{V}_0^+) \label{fwd1} \\
v_r(t)= |\tilde{V}_0^-|e^{\alpha z} \cos(\omega t + \beta z + \angle \tilde{V}_0^-) \label{bck1}
\end{eqnarray}


Figure \ref{fwrdref}  shows forward and reflected waves on a transmission line.  $z$ represents the length of the line,  and on the y-axis is the magnitude of the voltage.   The red line on both graphs is the voltage signal at a time  .1 ns. We would obtain  Figure \ref{fwrdref} if we had a camera that can take a picture of the voltage, and we took the first picture at $t_1=$ .1 ns on the entire transmission line.  The blue dotted line on both graphs is the same signal .1 ns later,  at time $t_2=$.2  ns.  We see that the signal has moved to the right in  1 ns,  from the generator to the load.  On the bottom graph, we see that at a time .1 ns, the red line represents the reflected signal.  The dashed blue line shows the signal at a time .2 ns. We see that the signal has moved to the left, from the load to the generator. 


\begin{figure}[ht!]
\begin{center}
\includegraphics[scale=0.5]{../jpg/frwrdwave_01.jpg}
\caption{\label{fwrdref} Forward (top) and reflected (bottom) waves on a transmission line.}
\end{center}
\end{figure}


\begin{verbatim}
clear all
clc
f = 10^9;
w = 2*pi*f
c=3*10^8;
beta=2*pi*f/c;
lambda=c/f;
t1=0.1*10^(-9)
t2=0.2*10^(-9)
x=0:lambda/20:2*lambda;

y1=sin(w*t1 - beta.*x);
y2=sin(w*t2 - beta.*x);
y3=sin(w*t1 + beta.*x);
y4=sin(w*t2 + beta.*x);

subplot (2,1,1),

    plot(x,y1,'r'),...
            hold on
    plot(x,y2,'--b'),...
    hold off
subplot (2,1,2),

    plot(x,y3,'r')
        hold on
    plot(x,y4,'--b')
    hold off
\end{verbatim}

Using Matlab code above, repeat the visualization of signals in the previous section for a lossy transmission line. Assume that $\alpha=0.1$\,Np, and all other variables are the same as in the previous section. How do the voltages compare in the lossy and lossless cases?

\end{example}





\end{document} 
