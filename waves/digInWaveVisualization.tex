\documentclass{ximera}  


\input{../preamble.tex}



 
\title{Visualization of lossless waves on transmission lines} 
\author{Milica Markovic} 
\outcome{Apply phasor transformation to a time-domain equation to obtain frequency-domain equation.}
\begin{document}  
\begin{abstract}  

\end{abstract}  
\maketitle    



\begin{example}

We will show next that if  the signs of the $\omega t$ and
$\beta z$ are the same, as in Equation \ref{bck1}, the wave moves in the forward $+z$
direction. If the signs of $\omega t$ and $\beta z$ are opposite, as in Equation \ref{fwd1}, the
wave moves in the $-z$ direction. In order to see this, we will visualize Equations \ref{fwd1} and \ref{bck1} using Matlab code below.

\begin{eqnarray}
v_f(t)=|\tilde{V}_0^+| e^{ - \alpha z} \cos(\omega t - \beta z + \angle \tilde{V}_0^+) \label{fwd1} \\
v_r(t)= |\tilde{V}_0^-|e^{\alpha z} \cos(\omega t + \beta z + \angle \tilde{V}_0^-) \label{bck1}
\end{eqnarray}


Figure \ref{fwrdref}  shows forward and reflected waves on a transmission line. On x-axis is the spatial coordinate $z$ from the generator to the load, where the transmission line is connected,  and on y-axis is   the magnitude of the voltage on the line.   The red line on both graphs is the voltage signal at a time  .1 ns. We would obtain  Figure \ref{fwrdref} if we had a camera that can take picture of voltages, and we took the first picture at $t_1=$ .1 ns on the entire transmission line.  The blue dotted line on both graphs is the same signal .1 ns later,  at time $t_2=$.2  ns.  We see that the signal has moved to the right in the time of 1 ns, or from the generator to the load.  On the bottom graph we see that at a time .1 ns, the red line represents the  reflected signal.  Dashed blue line shows the signal  at a time .2 ns. We see  that  the signal has moved to the left, or  from the load to the generator. 


\begin{figure}[ht!]
\begin{center}
\includegraphics[scale=0.5]{../jpg/frwrdwave_01.jpg}
\caption{\label{fwrdref} Forward (top) and reflected (bottom) waves on a transmission line.}
\end{center}
\end{figure}


\begin{verbatim}
clear all
clc
f = 10^9;
w = 2*pi*f
c=3*10^8;
beta=2*pi*f/c;
lambda=c/f;
t1=0.1*10^(-9)
t2=0.2*10^(-9)
x=0:lambda/20:2*lambda;

y1=sin(w*t1 - beta.*x);
y2=sin(w*t2 - beta.*x);
y3=sin(w*t1 + beta.*x);
y4=sin(w*t2 + beta.*x);

subplot (2,1,1),

    plot(x,y1,'r'),...
            hold on
    plot(x,y2,'--b'),...
    hold off
subplot (2,1,2),

    plot(x,y3,'r')
        hold on
    plot(x,y4,'--b')
    hold off
\end{verbatim}

Using matlab code above, repeat the visualization of signals  in the previous section for a lossy transmission line. Assume that $\alpha=0.1$\,Np, and all other variables are the same as in the previous section. How do the voltages compare in the lossy and lossless cases?

\end{example}





\end{document} 
