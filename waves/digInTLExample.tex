\documentclass{ximera}  


\input{../preamble.tex}



 
\title{Example Transmission Line Problem} 
\author{Milica Markovic} 
\outcome{Apply phasor transformation to a time-domain equation to obtain frequency-domain equation.}
\begin{document}  
\begin{abstract}  

\end{abstract}  
\maketitle    


\begin{example}

A transmitter operated at 20MHz, Vg=100V with $Z_g=50 \Omega$ internal impedance is connected to an antenna load through l=6.33m of the line. The line is a lossless $Z_0=50 \Omega$, $\beta=0.595rad/m$. The antenna impedance at 20MHz measures $Z_L=36+j20 \Omega$. Set the beginning of the z-axis at the load, as shown in Figure \ref{fig:TRLine}.
\begin{enumerate}
\item What is the electrical length of the line? 
\item What is the input impedance of the line $Z_{in}$?
\item What is the forward going voltage at the load $\tilde{V}_0^+$?
\item Find the expression for forward voltage anywhere on the line.
\item Find the expression for reflected voltage anywhere on the line.
\item Find the total voltage anywhere on the line.
\item Find the expression for forward current anywhere on the line.
\item Find the expression for reflected current anywhere on the line.
\item Find the total current anywhere on the line.
\item We now remove the antenna and connect load impedance $Z_L=50 \Omega$ to this 50 Ohm line, how will that change the equations above?
\end{enumerate}




\begin{figure}[htbp]
\begin{center}
\includegraphics[scale=0.3]{../jpg/trline.jpg}
%\strut\psfig{figure=trline.ps,width=3cm} \\
\end{center}
\caption{Transmission Line connects generator and the load.}
\label{fig:TRLine}
\end{figure}


\begin{explanation}

The equations for the voltage and current anywhere (any z) on a transmission line  are


\begin{eqnarray}
\tilde{V}(z)= \tilde{V}_0^+ (e^{-j \beta z} - |\Gamma_L|  e^{j \beta z + \Theta_r}  ) \label{eq:vtlfin} \\
I(z)=   \frac{\tilde{V}_0^+}{Z_0}  (e^{-j \beta z} - \Gamma_L  e^{j \beta z}  ) \label{eq:itlfin}
\end{eqnarray}

We are given phase constant $\beta$, and we have to find the other unknowns: phasor of voltage at the load $\tilde{V}_0^+$, and the reflection coefficient $\Gamma_L$.    

Since we know the load impedance $Z_L$, and the transmission-line impedance $Z_0$, we can find the reflection coefficient $\Gamma_L$ using Equation \ref{eq:reflcoe1}.

\begin{eqnarray}
\Gamma_L =  \frac{Z_L -Z_0}{Z_L +Z_0}=0.279 \, e^{j112^o}=0.279 \, e^{j1.95 \unit{rad}} \label{eq:reflcoe1}
\end{eqnarray}

To find the input impedance of the line, we use the equation. 


\begin{eqnarray}
Z_{in}= Z_0 \frac{Z_L+ j Z_0 \tan \beta l}{Z_0+ j Z_L \tan \beta l} = 70.8+ j 27.1 \Omega
\end{eqnarray}

We can use one of the following two equations to find the forward going voltage at the load:

\begin{eqnarray}
\tilde{V}_0^+= \frac{\tilde{V}_g Z_{in}}{Z_g + Z_{in}} \frac{1}{e^{j \beta l} + \Gamma_L e^{-j \beta l}} \\
\tilde{V}_0^+=V_g e^{-j \beta l} \frac{Z_0}{Z_0 (1+\Gamma_{in}) +Z_g (1-\Gamma_{in})}
\end{eqnarray}

Because the generator's impedance is equal to the transmission line impedance, we will use the second equation. When $Z_g=Z_0$ we see that the denominator simplifies into $Z_0 (1+\Gamma_{in}) +Z_g (1-\Gamma_{in}) = Z_0 (1+\Gamma_{in}+1\Gamma_{in})$ and we can further simplify the fraction to get the final value of $\tilde{V}_0^+=\frac{V_g}{2} e^{-j \beta l}$. Since $\beta l = \frac{2 \pi}{\lambda} \, 0.6 \lambda$, the forward going voltage at the load is $\tilde{V}_0^+=50 \, e^{-j1.2 \pi} = 50 \, e^{-j3.768} =50 \, e^{-j216^o} $.



The equations of the voltage and current anywhere on the line are therefore

\begin{eqnarray}
\tilde{V}(z)= 50  e^{-j3.768} (e^{-j 0.595 z} - 0.279  e^{j 0.595 z + 1.95}  ) \\
I(z)=   e^{-j3.768}   (e^{-j 0.595 z} - 0.279  e^{j 0.595 z+ 1.95}  ) 
\end{eqnarray}

If we replace the antenna with another load of impedance $50\Omega$, the reflection coefficient from the load will now be zero, and the reflected voltages will disappear, so the voltage and current will be 


\begin{eqnarray}
\tilde{V}(z)= \tilde{V}_0^+ e^{-j \beta z}   \\
I(z)=   \frac{\tilde{V}_0^+}{Z_0}  e^{-j \beta z} 
\end{eqnarray}


\end{explanation}




\end{example}






\end{document} 
