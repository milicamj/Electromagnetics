\documentclass{ximera}  


\input{../preamble.tex}



 
\title{Propagation constant and loss} 
\author{Milica Markovic} 
\outcome{Describe propagation constant. Describe current and voltage on a lossless transmission line.}
\begin{document}  
\begin{abstract}  

\end{abstract}  
\maketitle    





\section{Lossless transmission line}


In many practical applications, conductor loss is low $R\to 0$, and dielectric leakage is low $G \to 0$. These two conditions describe a
lossless transmission line.

In this case, the transmission line parameters are
\begin{itemize}
\item Propagation constant


\begin{eqnarray}
\gamma =\sqrt{(R+j\omega L)(G+ j\omega C)} \nonumber   \\ \nonumber
\gamma= \sqrt{j \omega L j \omega C} \\ \nonumber
\gamma = j \omega \sqrt{L C} = j \beta
\end{eqnarray}

\item Transmission line impedance will be defined in the next section, but it is also here for completeness.

\begin{eqnarray}
Z_0=\sqrt{\frac{R+j\omega L}{G+ j\omega C}} \nonumber  \\ \nonumber
Z_0=\sqrt{\frac{j\omega L}{ j\omega C}} \\ \nonumber
Z_0=\sqrt{\frac{L}{C}}
\end{eqnarray}

\item Wave velocity

\begin{eqnarray}
v=\frac{\omega}{\beta}  \nonumber \\ \nonumber
v=\frac{\omega}{\omega \sqrt{LC}} \\ \nonumber
v=\frac{1}{\sqrt{LC}}
\end{eqnarray}


\item Wavelength


\begin{eqnarray}
\lambda = \frac{2 \pi}{\beta}  \nonumber \\ \nonumber
\lambda = \frac{2 \pi}{ \omega \sqrt{LC}} \\ \nonumber
\lambda =\frac{2 \pi}{\sqrt{\epsilon_0 \mu_0 \epsilon_r} } \\ \nonumber
\lambda = \frac{c}{f \sqrt{\epsilon_r}} \\ \nonumber
\lambda = \frac{\lambda_0}{\sqrt{\epsilon_r}} \nonumber
\end{eqnarray}
\end{itemize}




\section{Voltage and current on lossless transmission line}

On a lossless transmission line, where $\gamma= j\beta$ current and voltage simplify to 

\begin{eqnarray}
\tilde{V}(z)=\tilde{V}_0^+ e^{-j \beta z} + \tilde{V}_0^- e^{j \beta z} \nonumber \\ \nonumber
\tilde{I}(z)=\tilde{I}_0^+ e^{- j \beta z} + \tilde{I}_0^- e^{j \beta z}
\end{eqnarray}


\section{What does it mean when we say a medium is lossy or lossless?}
In a lossless medium, electromagnetic energy is not turning into heat; there is no amplitude loss. An electromagnetic wave is heating a lossy material; therefore, the wave's amplitude decreases as $e^{-\alpha x}$.




\begin{center}
\begin{tabular}{|c|c|} \hline
medium     & attenuation constant $\alpha$ [dB/km]     \\  \hline       
coax        & 60                                 \\ \hline
 waveguide  & 2  \\ \hline          
fiber-optic &  0.5  \\ \hline
\end{tabular}
\end{center}


In guided wave systems such as transmission lines and waveguides, the attenuation of power with distance follows approximately $e^{-2\alpha x}$. The power radiated by an antenna falls off as $1/r^{2}$. As the distance between the source and load increases, there is a specific distance at which the cable transmission is lossier than antenna transmission.

\section{Low-Loss Transmission Line}

This section is optional.


In some practical applications, losses are small, but not negligible.  $R<< \omega L$ \footnote{metal resistance is
lower than the inductive impedance}and $G <<  \omega C$\footnote{dielectric conductance is lower than the capacitive impedance}. 

In this case, the transmission line parameters are
\begin{itemize}
\item Propagation constant

We can re-write the propagation constant as shown below. 
In somel applications,  losses are small, but not negligible.  $R<< \omega L$ and $G <<  \omega C$, then
in Equation \ref{lossytl2}, $ RG<< \omega^2 LC$.

\begin{eqnarray}
\gamma =\sqrt{(R+j\omega L)(G+ j\omega C)}   \\ 
\gamma= j\omega \sqrt{ L\, C}\sqrt{1\,-\,j\,\left( \frac{R}{\omega L}+\frac{G}{\omega C} \right)-\frac{R G}{\omega^2  L  C}} \label{lossytl2} \\ 
\gamma\approx j\omega \sqrt{ L\, C}\sqrt{1\,-\,j\,\left( \frac{R}{\omega L}+\frac{G}{\omega C} \right)}\label{lowtleq1}
\end{eqnarray}

Taylor's series for function $\sqrt{1+x}= \sqrt{1\,-\,j\,\left( \frac{R}{\omega L}+\frac{G}{\omega C} \right)}$ in Equation \ref{lowtleq1} is shown in Equations \ref{taylorser1}-\ref{taylorser2}.

\begin{eqnarray}
\sqrt{1+x}=1+\frac{x}{2}-\frac{x^2}{8}+\frac{x^3}{16}-...  \,for\, |x|<1 \label{taylorser1} \\
\gamma \approx  j\omega \sqrt{ L\, C} \sqrt{1\,-\,j\,\left( \frac{R}{\omega L}+\frac{G}{\omega C} \right)}= j\omega \sqrt{ L\, C}\left(1-\frac{j}{2} \left(  \frac{R}{\omega L}+\frac{G}{\omega C} \right)\right) \label{taylorser2}
\end{eqnarray}


The real and imaginary part of the propagation constant  $\gamma$ are:

\begin{eqnarray}
\alpha=  \frac{   \omega \sqrt{ L\, C}  }{2} \left(  \frac{R}{\omega L}+\frac{G}{\omega C} \right)  \\
\beta=     \omega \sqrt{ L\, C}
\end{eqnarray}


We see that the phase constant $\beta$ is the same as in the lossless case, and the attenuation constant $\alpha$ is frequency independent. All frequencies of a modulated signal are attenuated the same amount, and there is no dispersion on the line. When the phase constant is a linear function of frequency, $\beta=const \, \omega$, then the phase velocity is a constant $v_p=\frac{\omega}{\beta}=\frac{1}{const}$, and the group velocity is also a constant, and equal to the phase velocity. In this case, all frequencies of the modulated signal propagate at the same speed, and there is no distortion of the signal. 


\end{itemize}


\section{Transmission-line parameters R, G, C, and L}

To find the complex propagation constant $\gamma$, we need the transmission-line parameters R, G, C, and L. Equations for R, G, C, and L for a coaxial cable are given in the table below. 

\begin{center}
\begin{tabular}{|c|c|c|c|c|} \hline
Transmission-line    & R  & G & C &L     \\  \hline       
Coaxial Cable       & $\frac{R_{sd}}{2 \pi} \left(\frac{1}{a} +  \frac{1}{b} \right)$   & $\frac{2 \pi \sigma}{\ln b/a}$ &$\frac{2 \pi \epsilon}{\ln b/a}$& $\frac{\mu}{2 \pi } \ln b/a$                             \\ \hline    
\end{tabular}
\end{center}

Where $R_{sd} = \sqrt{\pi f \mu_m/\sigma_m}$ is the resistance associated with skin-depth. $f$ is the frequency of the signal, $\mu_m$ is the magentic permeability of conductors, $\sigma_c$ is the conductivity of conductors.

\begin{example}
Calculate capacitance per unit length of a coaxial cable if the inner radius is 0.02\,m, the outer radius is 0.06\,m, the dielectric constant is $\epsilon_r=2$. Use the applet below, Matlab, Matematica, or other software that you use.

\begin{center}  
\geogebra{whkrg2pu}{800}{600}  
\end{center} 

\end{example}



\end{document} 
